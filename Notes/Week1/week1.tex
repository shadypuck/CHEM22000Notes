\documentclass[../notes.tex]{subfiles}

\pagestyle{main}
\renewcommand{\chaptermark}[1]{\markboth{\chaptername\ \thechapter\ (#1)}{}}

\begin{document}




\chapter{The Basics: Bonding and Molecular Structure}
\section{Course Information}
\begin{itemize}
    \item \marginnote{9/28:}No labs this week.
    \item Virtual lab: Watch a video and record data in your notebook; answer embedded quiz questions.
    \item Collaborative Learning in Organic Chemistry (CLOC).
    \begin{itemize}
        \item 2hr Sunday or Monday.
        \item Contact Dr. Britni Ratliff (\href{mailto:ratliff@uchicago.edu}{ratliff@uchicago.edu}).
        \item Pass/Fail grading (based on attendance).
        \item You work on problems related to the lecture content under the supervision of someone who's taken the class before.
        \item You can opt-in/out on a quarter-by-quarter basis.
    \end{itemize}
    \item Review syllabus: Download alternate textbooks, put exam dates in the calendar, add office hours to calendar.
    \item Develop an understanding of how structure affects reactivity --- mechanistic principles.
    \item You don't have to memorize anything, but you have to remember everything.
    \begin{itemize}
        \item Like learning a language.
        \begin{itemize}
            \item Vocabulary, grammar (principles), apply to understand and predict.
        \end{itemize}
    \end{itemize}
\end{itemize}



\section{Defining Organic Chemistry}
\begin{itemize}
    \item \textbf{Organic chemistry}: Traditionally, the chemistry of living organisms. Now, the chemistry of carbon compounds.
    \begin{itemize}
        \item Carbon is of particular import because it can bond with itself, and it can form strong bonds with other elements (e.g., \ce{C}, \ce{O}, \ce{H}, \ce{S}, \ce{N}, and \ce{P}) as well.
        \item Carbon is bound in simple molecules (such as \ce{CO2} and \ce{CH4}), and highly complex ones (such as proteins, DNA, and RNA).
    \end{itemize}
    \item Carbon compounds:
    \begin{itemize}
        \item Natural: Sugars, fats, gasoline, hydrocarbons, hormones, natural drugs, peptides, rubber, silk, starch, cotton, etc.
        \item Synthetic: Dyes, fragrances, soaps, drugs, medicines, plastics, materials, teflon, nylon, etc.
    \end{itemize}
    \item OChem is a central science that feeds into fields such as biochemistry, molecular biology, molecular medicine, math/theory (e.g., buckyballs), engineering, and physics.
\end{itemize}



\section{Gen Chem Review}
\begin{itemize}
    \item Today:
    \begin{enumerate}
        \item Intro (done).
        \item Atomic structure and bonding (review from Gen Chem).
        \item Chemical bonds --- octet rule.
        \item Writing Lewis structures.
        \item Formal charges.
    \end{enumerate}
    \item Atomic structure and bonding.
    \begin{itemize}
        \item Atoms $\to$ elements $\to$ compounds.
        \item Nucleus (protons and neutrons) surrounded by electrons.
        \item This year, we'll concern ourselves with the main group elements.
        \item Electron configuration:
        \begin{itemize}
            \item Aufbau principle: Electrons fill orbitals from lowest energy to highest energy.
            \item Pauli exclusion principle: 2 elections/orbital with opposite spin quantum numbers (must pair $+\frac{1}{2}$ with $-\frac{1}{2}$).
            \item Hund's rule: Orbitals with equivalent energy get partially filled first before more electrons are added.
            \item Example: $1s^22s^22p^63s^1$ is \ce{Na}.
        \end{itemize}
        \item Valence electrons are key in this class.
    \end{itemize}
    \item Noble gas configurations and the octet rule.
    \begin{itemize}
        \item Lewis noticed that there is a special stability associated with a filled outer shell.
        \item Thus, we generally have 8 electrons in the filled outer shell.
        \begin{itemize}
            \item For example, \ce{Cl ->[1 e-] Cl-} and \ce{Na ->[][-1 e-] Na+}.
        \end{itemize}
        \item Chemical bonds form because they allow the atoms to achieve a filled octet.
        \item Two kinds of bonding: Ionic and covalent.
        \begin{itemize}
            \item Ionic: Not covered much this year. Lose or gain an electron (forming cations and anions, respectively) to for a filled outer shell. Usually involves a metal and a nonmetal.
            \item Covalent: Covered a lot this year. Sharing electrons to satisfy the need for an octet.
            \item The atoms involved dictate whether bonding will be ionic or covalent.
        \end{itemize}
        \item Electronegativity: The ability of an atom to attract its valence shell electrons.
        \begin{itemize}
            \item Defined by Pauling, who let $\ce{Li}=1.0$ and $\ce{F}=4.0$.
            \item This is a very important concept for understanding bonding and reactivity.
            \item EN increases across and up on the periodic table: More protons and a shorter distance away from the nucleus both mean a greater pull on the electrons.
            \item Mnemonic (highest to lowest electronegativity): \ce{F} \ce{O} \ce{Cl} \ce{N} \ce{Br} \ce{I} \ce{S} \ce{C} \ce{H} \ce{P}.
            \item Non-polar covalent bonds form when $\Delta\text{EN}<0.5$.
            \item Polar covalent bonds form when $\Delta\text{EN}\approx 0.5-1.9$.
        \end{itemize}
        \item Exceptions to the octet rule: \ce{H} wants $2\,\e[-]$. \ce{Be} wants $4\,\e[-]$. \ce{B} and \ce{Al} want $6\,\e[-]$. Molecule has an odd number of electrons (e.g., \ce{NO} with 11 electrons is stable).
    \end{itemize}
    \item Lewis structures.
    \begin{itemize}
        \item General rules/procedure (there are exceptions).
        \begin{enumerate}
            \item Determine the total number of valence electrons for the molecule. Add electrons for negative charges; remove for positive charges.
            \item Draw a skeleton and join atoms with single bonds. Put the atom that likes to make the most bonds in the center.
            \item Deduct 2 electrons from the count in step 1 for each single bond. Fill outside atoms with lone pair electrons.
            \item The remaining electrons go on the central atom.
            \item If you have too few electrons for every atom to have an octet, use lone pair electrons to convert single bonds to double bonds. We can also use triple bonds.
        \end{enumerate}
        \item \ce{CH4} and \ce{NH3} presented as a worked example.
    \end{itemize}
\end{itemize}




\end{document}