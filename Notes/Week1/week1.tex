\documentclass[../notes.tex]{subfiles}

\pagestyle{main}
\renewcommand{\chaptermark}[1]{\markboth{\chaptername\ \thechapter\ (#1)}{}}

\begin{document}




\chapter{The Basics: Bonding and Molecular Structure}
\section{Course Information}
\begin{itemize}
    \item \marginnote{9/28:}No labs this week.
    \item Virtual lab: Watch a video and record data in your notebook; answer embedded quiz questions.
    \item Collaborative Learning in Organic Chemistry (CLOC).
    \begin{itemize}
        \item 2hr Sunday or Monday.
        \item Contact Dr. Britni Ratliff (\href{mailto:ratliff@uchicago.edu}{ratliff@uchicago.edu}).
        \item Pass/Fail grading (based on attendance).
        \item You work on problems related to the lecture content under the supervision of someone who's taken the class before.
        \item You can opt-in/out on a quarter-by-quarter basis.
    \end{itemize}
    \item Review syllabus: Download alternate textbooks, put exam dates in the calendar, add office hours to calendar.
    \item Develop an understanding of how structure affects reactivity --- mechanistic principles.
    \item You don't have to memorize anything, but you have to remember everything.
    \begin{itemize}
        \item Like learning a language.
        \begin{itemize}
            \item Vocabulary, grammar (principles), apply to understand and predict.
        \end{itemize}
    \end{itemize}
\end{itemize}



\section{Defining Organic Chemistry}
\begin{itemize}
    \item \textbf{Organic chemistry}: Traditionally, the chemistry of living organisms. Now, the chemistry of carbon compounds.
    \begin{itemize}
        \item Carbon is of particular import because it can bond with itself, and it can form strong bonds with other elements (e.g., \ce{C}, \ce{O}, \ce{H}, \ce{S}, \ce{N}, and \ce{P}) as well.
        \item Carbon is bound in simple molecules (such as \ce{CO2} and \ce{CH4}), and highly complex ones (such as proteins, DNA, and RNA).
    \end{itemize}
    \item Carbon compounds:
    \begin{itemize}
        \item Natural: Sugars, fats, gasoline, hydrocarbons, hormones, natural drugs, peptides, rubber, silk, starch, cotton, etc.
        \item Synthetic: Dyes, fragrances, soaps, drugs, medicines, plastics, materials, teflon, nylon, etc.
    \end{itemize}
    \item OChem is a central science that feeds into fields such as biochemistry, molecular biology, molecular medicine, math/theory (e.g., buckyballs), engineering, and physics.
\end{itemize}



\section{Gen Chem Review}
\begin{itemize}
    \item Today:
    \begin{enumerate}
        \item Intro (done).
        \item Atomic structure and bonding (review from Gen Chem).
        \item Chemical bonds --- octet rule.
        \item Writing Lewis structures.
        \item Formal charges.
    \end{enumerate}
    \item Atomic structure and bonding.
    \begin{itemize}
        \item Atoms $\to$ elements $\to$ compounds.
        \item Nucleus (protons and neutrons) surrounded by electrons.
        \item This year, we'll concern ourselves with the main group elements.
        \item Electron configuration:
        \begin{itemize}
            \item Aufbau principle: Electrons fill orbitals from lowest energy to highest energy.
            \item Pauli exclusion principle: 2 elections/orbital with opposite spin quantum numbers (must pair $+\frac{1}{2}$ with $-\frac{1}{2}$).
            \item Hund's rule: Orbitals with equivalent energy get partially filled first before more electrons are added.
            \item Example: $1s^22s^22p^63s^1$ is \ce{Na}.
        \end{itemize}
        \item Valence electrons are key in this class.
    \end{itemize}
    \item Noble gas configurations and the octet rule.
    \begin{itemize}
        \item Lewis noticed that there is a special stability associated with a filled outer shell.
        \item Thus, we generally have 8 electrons in the filled outer shell.
        \begin{itemize}
            \item For example, \ce{Cl ->[1 e^-] Cl-} and \ce{Na ->[][-1 e^-] Na+}.
        \end{itemize}
        \item Chemical bonds form because they allow the atoms to achieve a filled octet.
        \item Two kinds of bonding: Ionic and covalent.
        \begin{itemize}
            \item Ionic: Not covered much this year. Lose or gain an electron (forming cations and anions, respectively) for a filled outer shell. Usually involves a metal and a nonmetal.
            \item Covalent: Covered a lot this year. Sharing electrons to satisfy the need for an octet.
            \item The atoms involved dictate whether bonding will be ionic or covalent.
        \end{itemize}
        \item Electronegativity: The ability of an atom to attract its valence shell electrons.
        \begin{itemize}
            \item Defined by Pauling, who let $\ce{Li}=1.0$ and $\ce{F}=4.0$.
            \item This is a very important concept for understanding bonding and reactivity.
            \item EN increases across and up on the periodic table: More protons and a shorter distance away from the nucleus both mean a greater pull on the electrons.
            \item Mnemonic (highest to lowest electronegativity): \ce{F} \ce{O} \ce{Cl} \ce{N} \ce{Br} \ce{I} \ce{S} \ce{C} \ce{H} \ce{P}.
            \item Non-polar covalent bonds form when $\Delta\text{EN}<0.5$.
            \item Polar covalent bonds form when $\Delta\text{EN}\approx 0.5-1.9$.
        \end{itemize}
        \item Exceptions to the octet rule: \ce{H} wants $2\,\e[-]$. \ce{Be} wants $4\,\e[-]$. \ce{B} and \ce{Al} want $6\,\e[-]$. Molecule has an odd number of electrons (e.g., \ce{NO} with 11 electrons is stable).
    \end{itemize}
    \item Lewis structures.
    \begin{itemize}
        \item General rules/procedure (there are exceptions).
        \begin{enumerate}
            \item Determine the total number of valence electrons for the molecule. Add electrons for negative charges; remove for positive charges.
            \item Draw a skeleton and join atoms with single bonds. Put the atom that likes to make the most bonds in the center.
            \item Deduct 2 electrons from the count in step 1 for each single bond. Fill outside atoms with lone pair electrons.
            \item The remaining electrons go on the central atom.
            \item If you have too few electrons for every atom to have an octet, use lone pair electrons to convert single bonds to double bonds. We can also use triple bonds.
        \end{enumerate}
        \item \ce{CH4} and \ce{NH3} presented as worked examples.
    \end{itemize}
    \item \marginnote{9/30:}Today:
    \begin{enumerate}
        \setcounter{enumi}{3}
        \item Lewis Structures.
        \item Formal charges.
        \item Isomers.
        \item Structural formulas.
        \item Resonance.
        \item Orbitals and bonding.
    \end{enumerate}
    \item Lewis structures:
    \begin{itemize}
        \item \ce{H2CO} (formaldehyde) and \ce{CH3COOH} (acetic acid) presented as worked examples.
    \end{itemize}
    \item Formal charge determination:
    \begin{itemize}
        \item If the number of valence electrons does not equal the total number of electrons on an atom, then you will have a formal charge.
        \item Rule:
        \begin{align*}
            \text{Formal Charge} &= \text{normal valence }\e[-]-\text{actual }\e[-]\\
            &= \text{valence }\e[-]-\left( \text{nonbonding }\e[-]+\frac{1}{2}\text{ bonding }\e[-] \right)\\
            &= \text{valence }\e[-]-(\text{dots}+\text{lines})
        \end{align*}
        \item \ce{CH3COO-} (acetate) has a formal charge of $6-7=-1$ on its singly bonded oxygen.
        \item \ce{CH3NH3+} (methyl ammonium) has a formal charge of $5-4=+1$ on its nitrogen.
        \item Exceptions: Open shell Group III central atoms (e.g., \ce{B} and \ce{Al}).
        \begin{itemize}
            \item \ce{BF3} acts as a Lewis acid because it wants to grab $2\,\e[-]$ to form an octet.
            \item It often acts in acid-base coupling reactions, grabbing a lone pair from an oxygen in an adjacent molecule and bonding through it.
        \end{itemize}
    \end{itemize}
\end{itemize}



\section{OChem Basics}
\begin{itemize}
    \item Isomers:
    \begin{itemize}
        \item Constitutional isomers: Same molecular formula but different bond connectivities.
        \item Acetone vs. 3-propenol, yet both are \ce{C3H6O}.
    \end{itemize}
    \item Structural formulas:
    \begin{figure}[h!]
        \centering
        \footnotesize
        \begin{subfigure}[b]{0.3\linewidth}
            \centering
            \chemfig{H-C(-[2]H)(-[6]H)-C(-[2]H)(-[6]H)-C(-[2]Cl)(-[6]H)-C(-[2]H)(-[6]H)-H}
            \caption{Dash structural formula.}
            \label{fig:structuralFormulasa}
        \end{subfigure}
        \begin{subfigure}[b]{0.3\linewidth}
            \centering
            \chemfig{-[:-30]-[:30](-[2]Cl)-[:-30]}
            \caption{Bond line formula.}
            \label{fig:structuralFormulasb}
        \end{subfigure}
        \begin{subfigure}[b]{0.3\linewidth}
            \centering
            \ce{CH3CH3CH2ClCH3}
            \caption{Condensed formula.}
            \label{fig:structuralFormulasc}
        \end{subfigure}
        \caption{Structural formulas.}
        \label{fig:structuralFormulas}
    \end{figure}
    \item \textbf{Dash structural formula}: A Lewis structure.
    \item \textbf{Bond line formula}: No \ce{C-H}'s, show a vertex for each carbon, show heteroatoms and heteroatom \ce{H}'s. \emph{Also known as} \textbf{line-angle structure}, \textbf{zig-zag structure}.
    \item \textbf{Condensed formula}: All atoms written out with no bonds or lone pairs.
    \item \textbf{3D representation}: A dash structural/bond line formula with wedges and dashes.
    \item \textbf{Resonance}: When a molecule or an ion can be represented by 2 or more Lewis structures, i.e., two or more structures with the same skeleton connected by different electrons.
    \begin{itemize}
        \item Resonance structures or resonance contributors.
        \item The actual molecule is somewhere between the contributors.
        \item \ce{CO3^2-} presented as a worked example.
        \item Guidelines:
        \begin{enumerate}
            \item Only lone pairs or $\pi$ electrons move (never move single bonds).
            \item No structure with greater than $8\,\e[-]$ on a 2nd row atom.
            \item The species with the maximum number of octets is the strongest contributor.
            \item Charge on suitable atoms (e.g., negative charge on the atom with the highest electronegativity).
        \end{enumerate}
        \item Resonance stabilization comes from delocalization. When 2 or more resonance structures, the "real" structure is somewhere in between (the real is more stable than any contributor).
        \item \ce{CH3COO-} (acetate), \ce{CH2CHCH2+}, and \ce{(CH3)2CO} presented as worked examples.
        \item You can also depict delocalization with a curving dashed bond and $\delta^-$'s.
    \end{itemize}
\end{itemize}



\section{Bonding and Orbital Diagrams}
\begin{itemize}
    \item \marginnote{10/5:}Today:
    \begin{enumerate}
        \setcounter{enumi}{8}
        \item Orbital theory and bonding.
        \item Methane.
        \item Ethane.
        \item Ethylene.
        \item Acetylene.
        \item Comparison of $sp^3$, $sp^2$, $sp$ orbitals.
        \item VSEPR Model + Molecular Symmetry.
    \end{enumerate}
    \item Orbital theory and bonding:
    \begin{itemize}
        \item Defines \textbf{atomic orbitals}.
        \item Reviews $s$ and $p$ orbital shapes, positive and negative regions, and nodes.
        \item Energy of orbitals diagram.
        \item Phosphorous and sulfur can exceed the octet rule since they have $d$ orbitals in which to stash extra electrons.
        \item Filled with the Aufbau/Pauli Exclusion principles, and Hund's Rule.
        \item Goes over bonding energy diagram.
        \item Mathematically, we have a Linear Combination of Atomic Orbitals (or LCAO).
        \begin{itemize}
            \item Electrons are represented as waves; thus, they have $+$ and $-$ phases.
            \item Opposite phases are destructive; this forms $\sigma^*$ orbitals.
            \item Same phases are constructive; this forms $\sigma$ orbitals.
        \end{itemize}
        \item Goes over MO diagrams.
    \end{itemize}
    \item \textbf{Atomic orbital}: A space where electrons are likely to be found 95\% of the time.
    \item \textbf{Degenerate} (orbitals): Two orbitals with the same energy.
    \item \textbf{Chemical bond}: A favorable interaction between 2 atoms, i.e., one that helps to fill the outer orbitals to achieve a noble gas configuration.
    \item Bonding in methane:
    \begin{figure}[h!]
        \centering
        \begin{tikzpicture}
            \footnotesize
    
            \begin{scope}
                \draw [thick,-latex] (0,0) -- node[left]{$E$} (0,2.5);
                \draw (0.8,0.2) -- node[above]{$\upharpoonleft\downharpoonright$} ++(0.5,0) node[right=6mm]{$1s$};
                \draw (0.8,1) -- node[above]{$\upharpoonleft\downharpoonright$} ++(0.5,0) node[right=6mm]{$2s$};
                \draw (0.2,1.5) -- node[above]{$\upharpoonleft$} ++(0.5,0);
                \draw (0.8,1.5) -- node[above]{$\upharpoonleft$} ++(0.5,0);
                \draw (1.4,1.5) -- ++(0.5,0) node[right]{$2p$};
            \end{scope}
            \node at (3.5,1.5) {\ce{->[promotion]}};
            \begin{scope}[xshift=5cm]
                \draw [thick,-latex] (0,0) -- node[left]{$E$} (0,2.5);
                \draw (0.8,0.2) -- node[above]{$\upharpoonleft\downharpoonright$} ++(0.5,0) node[right=6mm]{$1s$};
                \draw (0.8,1) -- node[above]{$\upharpoonleft$} ++(0.5,0) node[right=6mm]{$2s$};
                \draw (0.2,1.5) -- node[above]{$\upharpoonleft$} ++(0.5,0);
                \draw (0.8,1.5) -- node[above]{$\upharpoonleft$} ++(0.5,0);
                \draw (1.4,1.5) -- node[above]{$\upharpoonleft$} ++(0.5,0) node[right]{$2p$};
            \end{scope}
            \node at (8.5,1.5) {\ce{->[hybridization]}};
            \begin{scope}[xshift=10cm]
                \draw [thick,-latex] (0,0) -- node[left]{$E$} (0,2.5);
                \draw (1.1,0.2) -- node[above]{$\upharpoonleft\downharpoonright$} ++(0.5,0) node[right=12mm]{$1s$};
                \draw (0.2,1.35) -- node[above]{$\upharpoonleft$} ++(0.5,0);
                \draw (0.8,1.35) -- node[above]{$\upharpoonleft$} ++(0.5,0);
                \draw (1.4,1.35) -- node[above]{$\upharpoonleft$} ++(0.5,0);
                \draw (2,1.35) -- node[above]{$\upharpoonleft$} ++(0.5,0) node[right]{$2sp^3$};
            \end{scope}
        \end{tikzpicture}
        \caption{Bonding in methane.}
        \label{fig:methaneEnergyDiagram}
    \end{figure}
    \begin{itemize}
        \item Draws an orbital diagram for carbon.
        \item Promotes an electron from $2s\to 2p_z$.
        \item Hybridizes $2s,2p_x,2p_y,2p_z$ into 4 degenerate $sp^3$ orbitals of weighted average energy, each containing only 1 electron.
        \item Links each of these $sp^3$ electrons to the $1s$ electron in \ce{H2}, forming $\sigma$ orbitals.
        \item The new orbitals adopt a tetrahedral arrangement to be as far apart as possible.
    \end{itemize}
    \item Bonding in ethane.
    \begin{itemize}
        \item Two $sp^3$ electrons combine in a $\sigma$ orbital; no electrons go into the $\sigma^*$ MO.
    \end{itemize}
    \item The structure of ethylene.
    \begin{itemize}
        \item Side by side overlap of $p$ orbitals forms a $\pi$ bond.
        \item The angle between the hydrogens in ethylene is slightly less than $\ang{120}$.
        \item The bond is slightly shorter than in ethane (greater $s$ character plus an additional type of bond).
        \item Features of the \ce{C=C} double bond.
        \begin{itemize}
            \item $sp^2$-hybridized carbons making $3\sigma$ and $1\pi$ bond.
            \item A $\pi$ bond is weaker than a $\sigma$ bond, but still strong.
            \item $\sigma_{sp^2-sp^2}$ is stronger than $\sigma_{sp^3-sp^3}$.
            \item Restricted rotation (hard to twist \ce{C2H2} by $\ang{90}$).
            \item \emph{cis}-\emph{trans} isomerism as a result of restricted rotation.
            \item The $\pi$ bond acts like a Lewis base with some systems since the $\pi$ electrons are held relatively weakly. In other words, the $\pi$ electrons are exposed.
        \end{itemize}
        \item Draws an MO diagram for the carbons.
    \end{itemize}
    \item The structure of acetylene.
    \begin{itemize}
        \item 2 $\pi$ bonds, 1 $\sigma$ bond.
        \item Even greater strength, but not quite as much greater as the $\sigma_{sp^3-sp^3}\to\sigma_{sp^2-sp^2}$ difference.
    \end{itemize}
\end{itemize}



\section{VSEPR Theory}
\begin{itemize}
    \item \marginnote{10/7:}There's a special kind of electronegativity that relates to hybridization: An $sp$-hybridized carbon is more electronegative than an $sp^3$-hybridized carbon, for instance.
    \item VSEPR Model:
    \begin{itemize}
        \item Electron pairs want to stay as far apart as possible in space.
        \item Consider the bonding electrons (number of atoms bound) and nonbonding electrons.
        \item Describe shape based on the position of nuclei.
    \end{itemize}
    \item Constructs VSEPR table for linear, trigonal planar, tetrahedral, trigonal pyramidal, bent.
\end{itemize}




\end{document}