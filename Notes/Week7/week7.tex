\documentclass[../notes.tex]{subfiles}

\pagestyle{main}
\renewcommand{\chaptermark}[1]{\markboth{\chaptername\ \thechapter\ (#1)}{}}
\setcounter{chapter}{6}

\begin{document}




\chapter{Reactions of Alkynes}
\section{Methods of Hydration}
\begin{itemize}
    \item \marginnote{11/9:}Mukaiyama Hydration.
    \begin{itemize}
        \item A "greener" method.
    \end{itemize}
    \item General form.
    \begin{equation*}
        \ce{R-= ->[Co(acac)2, O2][PhSiH3] R-(-OH)-}
    \end{equation*}
    \item Mechanism:
    \begin{figure}[h!]
        \centering
        \footnotesize
        \begin{subfigure}[b]{\linewidth}
            \centering
            \chemfig{Co?(-[:35]O=[::-35](-[::60])-[::-60]=_[::-60](-[::60])-[::-60]O?)(-[:215]O=[::-35](-[::60])-[::-60]=_[::-60](-[::60])-[::-60]O?)}
            \caption{Cobalt catalyst.}
            \label{fig:mukaiyamaHydrationa}
        \end{subfigure}\\[2em]
        \begin{subfigure}[b]{\linewidth}
            \centering
            \schemestart
                \chemfig{R-[:30]=_[:-30]}
                \arrow{->[\footnotesize[\ce{Co-H}]]}[,1.3]
                \chemfig{R-[:30](-[2]{[Co]})-[:-30]-[:30]H}
                \arrow{->[\footnotesize\ce{O2}]}
                \chemfig{R-[:30](-[2]O-[:30]O-[2]{[Co]})-[:-30]-[:30]H}
                \arrow{->[\footnotesize\ce{PhSiH3}]}[,1.3]
                \chemfig{R-[:30](-[2]OH)-[:-30]-[:30]H}
            \schemestop
            \caption{Mechanism.}
            \label{fig:mukaiyamaHydrationb}
        \end{subfigure}
        \caption{Mukaiyama hydration mechanism.}
        \label{fig:mukaiyamaHydration}
    \end{figure}
    \begin{itemize}
        \item Note that \ce{Co(acac)2} is a catalyst.
    \end{itemize}
    \item Hydroboration/oxidation of alkene.
    \begin{itemize}
        \item Nobel Prize (1979).
        \item Discovered by a UChicago undergrad, H. C. Brown.
    \end{itemize}
    \item General form.
    \begin{equation*}
        \ce{R-= ->[1. BH3][2. H2O2, NaOH] R---OH}
    \end{equation*}
    \begin{itemize}
        \item Regioselective (anti-Markovnikov).
        \item Stereospecific (\ce{H} and \ce{OH} add cis).
        \item Covers \ce{BH3} with its empty $p$ orbital that makes it a good Lewis acid and dimerizes to \ce{B2H6} with its "3C-2e" bond in the gas phase.
    \end{itemize}
    \item Mechanism:
    \begin{figure}[h!]
        \centering
        \footnotesize
        \begin{subfigure}[b]{\linewidth}
            \centering
            \schemestart
                \chemfig{R-[:30]@{C1}=_[@{db}:-30]}
                \+
                \chemfig{H-[@{sb}]@{B2}B(<:[:20]H)(<[:-20]H)}
                \arrow
                \chemleft{[}
                    \chemfig{R-[:30]@{C3a}\phantom{}?-[:-30,1.3]@{C3b}-[:60,1.1,,,dashed]B(<[:-20]H)(<:[:20]H)-[:150,0.9,,,dashed]H?[,,dashed]}
                \chemright{]^\ddagger}
                \arrow
                \chemfig{R-[:30](-[2]H)-[:-30]-[:30]B(<:[:20]H)(<[:-20]H)}
                % \arrow{->[\footnotesize 2 \chemfig[atom sep=1.4em]{R-[:30]=_[:-30]}]}[,1.5]
                % \chemfig{@{R5}R-[:30]-[:-30]@{C5}-[:30]B}
            \schemestop
            \chemmove{
                \draw [rex,semithick,shorten <=4pt,shorten >=2pt] (db) to[out=-120,in=-90] (B2);
                \draw [rex,semithick,shorten <=2pt,shorten >=2pt] (sb) to[out=90,in=90] (C1);
                \draw [dashed,-] ([xshift=1pt,yshift=2pt]C3a.center) -- ([xshift=1pt,yshift=2pt]C3b.center);
                % \path (R5) node[left]{$\bigg($} -- (C5) node[right]{$\bigg)_3$};
            }
            \caption{Part I.}
            \label{fig:hydroborationa}
        \end{subfigure}\\[2em]
        \begin{subfigure}[b]{\linewidth}
            \centering
            \schemestart
                \chemfig{R-[:30]-[:-30]-[:30]@{B1}B-[:-30]@{C1}-[:30]-[:-30]@{R1}R}
                \+{1.5em}
                \chemfig{@{O2}\charge{90=\:,135:3pt=$\ominus$}{O}-O-H}
                \arrow
                \chemfig{R-[:30]-[:-30]-[@{sb1}:30]\charge{45:3pt=$\ominus$}{B}(-[2]@{O3a}O-[@{sb2}:30]@{O3b}O-[2]H)-[:-30]@{C3}-[:30]-[:-30]@{R3}R}
                \arrow{->[][-\ce{OH-}]}
                \chemfig{R-[:30]-[:-30]-[:30]O-[:-30]B-[:30]@{C4}-[:-30]-[:30]@{R4}R}
            \schemestop
            \chemmove{
                \path (C1) node[left]{$\bigg($} -- (R1) node[right]{$\bigg)_2$};
                \draw [rex,semithick,shorten <=6pt,shorten >=2pt] (O2) to[out=90,in=90] (B1);
                \path (C3) node[left]{$\bigg($} -- (R3) node[right]{$\bigg)_2$};
                \draw [rex,semithick,shorten <=2pt,shorten >=2pt] (sb1) to[bend left=30] (O3a);
                \draw [rex,semithick,shorten <=2pt,shorten >=2pt] (sb2) to[bend left=60,looseness=2] (O3b);
                \path (C4) node[left]{$\bigg($} -- (R4) node[right]{$\bigg)_2$};
            }
            \caption{Part II: Oxidative cleavage of the \ce{B-C} bond.}
            \label{fig:hydroborationb}
        \end{subfigure}\\[2em]
        \begin{subfigure}[b]{\linewidth}
            \centering
            \schemestart
                \chemfig{R{(}CH_2{)}_2O-[:30]@{B1}B-[:-30]{[}O{(}CH_2{)}_2R{]}_2}
                \arrow{<->>[\footnotesize\chemfig[atom sep=1.4em]{@{O2}\charge{90=\:,135:3pt=$\ominus$}{O}-H}]}[,1.2]
                \chemfig{R{(}CH_2{)}_2@{O3}O-[@{sb3}:30]\charge{45:3pt=$\ominus$}{B}(-[2]O-[:30]H)-[:-30]{[}O{(}CH_2{)}_2R{]}_2}
                \arrow{<=>}
                \chemfig{R{(}CH2{)}_2@{O4}\charge{90=\:,45:3pt=$\ominus$}{O}}
                \+
                \chemfig{@{H5}H-[@{sb5}]@{O5}O-B-{[}O{(}CH_2{)}_2R{]}_2}
                \arrow{->}[-90]
                \chemfig{R{(}CH2{)}_2OH}
                \+
                \chemfig{{(}\charge{45=$\oplus$}{Na}{)}\hspace{1em}\charge{135=$\ominus$}{O}-B-{[}O{(}CH_2{)}_2R{]}_2}
            \schemestop
            \chemmove{
                \draw [rex,semithick,shorten <=6pt,shorten >=2pt] (O2) to[bend right=90] (B1);
                \draw [rex,semithick,shorten <=2pt,shorten >=2pt] (sb3) to[bend right=60,looseness=2] (O3);
                \draw [rex,semithick,shorten <=6pt,shorten >=2pt] (O4) to[bend left=90,looseness=2] (H5);
                \draw [rex,semithick,shorten <=2pt,shorten >=2pt] (sb5) to[bend left=90,looseness=3] (O5);
            }
            \caption{Part III: Hydrolysis of boric acid ester.}
            \label{fig:hydroborationc}
        \end{subfigure}
        \caption{Hydroboration mechanism.}
        \label{fig:hydroboration}
    \end{figure}
    \begin{itemize}
        \item The step in Part I is a concerted step (this is key for cis addition).
        \item \ce{BH3} adds the way it does in part I (with the \ce{H} put on the more substituted carbon) due to sterics (bulky group attaches to the less bulky side) and electronics (the $\pi$-bonding electrons of the alkene connecting to the boron first will create a partial positive charge, and it is more stable to have that partial positive on the more substituted position).
        \item The alkene does not attack an \ce{H} on \ce{BH3} because said \ce{H}'s are not acidic (it is the boron that is electron deficient).
        \item Each part repeats an additional two times with the product of the $n^\text{th}$ run the reactant of the $(n+1)^\text{th}$ run to create the reactant of the next part.
        \item In part II, mixing \ce{H2O2} with \ce{OH-} yields a deprotonated peroxide (\ce{O2H-}).
        \item The final product of part II is the boric acid ester.
        \item Part III has a borate salt as a final byproduct.
        \item The mechanism implies that the hydrogen added comes from \ce{BH3}, and the oxygen added (as part of the hydroxide) comes from \ce{H2O2}.
    \end{itemize}
    \item If we add to an alkene borane, and then a strong acid and heat, we end up hydrogenating it.
    \item Summary of alcohol synthesis:
    \begin{itemize}
        \item If you want Markovnikov addition, use\dots
        \begin{itemize}
            \item Acid-catalyzed hydration.
            \item Oxymercuration/reduction.
            \item Mukaiyama hydration.
        \end{itemize}
        \item If you want anti-Markovnikov addition, use\dots
        \begin{itemize}
            \item Hydroboration.
        \end{itemize}
    \end{itemize}
    \item Ozonolysis of alkenes.
    \begin{itemize}
        \item An \textbf{oxidation} reaction (adding oxygen or removing hydrogen).
        \item Treat an alkene with ozone and dimethyl sulfide to cleave the \ce{C=C} bond and add oxygen onto each carbon, forming carbonyls (and associated groups).
        \item Note that if we consider the resonance structures of ozone, we will find that charger separation is unavoidable, i.e., that the molecule must have a plus and a minus charge somewhere. This makes it very reactive.
    \end{itemize}
    \item Mechanism.
    \begin{figure}[h!]
        \centering
        \footnotesize
        \schemestart
            \chemfig{R>:[:70]@{C1}(<[:-70]H)=_[@{db1}2](<[:70]H)<:[:110]R}
            \arrow{0}[,0.5]
            \chemfig{@{O2a}\charge{0=\:,-90=\:,180=\:}{O}-[:60]@{O2b}\charge{0=\:}{O}=[@{db2}:120]@{O2c}\charge{[extra sep=1.5pt]45=\:,135=\:}{O}}
            \arrow
            \chemname{\chemfig[atom sep=3em]{*5((<:[:-150,0.7]R)(<[:-110,0.7]H)-[@{sb3a}]\charge{[extra sep=1.5pt]-45=\:,-135=\:}{O}-[@{sb3b}]@{O3b}\charge{[extra sep=1.5pt]45=\:,-45=\:}{O}-@{O3c}\charge{[extra sep=1.5pt]45=\:,135=\:}{O}-[@{sb3c}](<:[:150,0.7]R)(<[:110,0.7]H)-[@{sb3d}])}}{Molozonide}
            \arrow
            \chemfig{R-[:30]@{C4}(=[@{db4}2]@{O4}\charge{[extra sep=1.5pt]45=\:,135=\:}{O})-[:-30]H}
            \arrow{0}[,0]\+{,,-2em}
            \chemfig{R-[:-30]@{C5}(=[@{db5}6]@{O5a}\charge{[extra sep=1.5pt]-45=\:}{O}-[:-150]@{O5b}\charge{90=\:,180=\:,-90=\:}{O})-[:30]H}
            \arrow
            \chemname{\chemfig[atom sep=3em]{[:18]*5(@{O6a}\charge{-90=\:,180=\:}{O}?-[@{sb6}]@{O6b}\charge{0=\:,-90=\:}{O}-(<:[:38,0.7]R)(<[:-2,0.7]H)-\charge{[extra sep=1.5pt]45=\:,135=\:}{O}-?(<[:182,0.7]H)(<:[:142,0.7]R))}}{Ozonide}
            \arrow{->[*{0}{\chemfig[atom sep=1.4em]{-[:30]@{S7}\charge{90=\:,-90=\:}{S}-[:-30]}}]}[-90]
            \chemfig{@{O8a}\charge{0=\:,-90=\:,180=\:,45:3pt=$\ominus$}{O}-[@{sb8a}2](<[:150]H)(<:[:110]R)-[@{sb8b}:30]\charge{[extra sep=1.5pt]45=\:,135=\:}{O}-[@{sb8c}:-30](<[:30]H)(<:[:70]R)-[@{sb8d}6]\charge{180=\:,-90=\:}{O}-[@{sb8e}:-30]\charge{90=\:,45:3pt=$\oplus$}{S}(-[6])-[:30]}
            \arrow[180]
            \subscheme{
                \chemfig{H-[:-30](=[6]\charge{[extra sep=1.5pt]-45=\:,-135=\:}{O})-[:30]R}
                \arrow{0}[,0]\+{,,2em}
                \chemfig{H-[2](=[:150]\charge{90=\:,180=\:}{O})-[:30]R}
                \arrow(.east--.160){0}[,0]\+{,,2em}
                \chemname[-1em]{\chemfig{-[:30]\charge{-90=\:}{S}(=[2]\charge{[extra sep=1.5pt]45=\:,135=\:}{O})-[:-30]}}{DMSO}
            }
        \schemestop
        \chemmove{
            \draw [rex,semithick,shorten <=4pt,shorten >=2pt] (db1) to[bend right=30] (O2c);
            \draw [rex,semithick,shorten <=3pt,shorten >=2pt] (db2) to[bend left=60,looseness=2.5] (O2b);
            \draw [rex,semithick,shorten <=5pt,shorten >=4pt] (O2a) to[bend left=15] (C1);
            %
            \draw [rex,semithick,shorten <=2pt,shorten >=2pt] (sb3d) to[bend left=30,looseness=1.2] (sb3a);
            \draw [rex,semithick,shorten <=2pt,shorten >=6pt] (sb3b) to[bend right=90,looseness=3] (O3b);
            \draw [rex,semithick,shorten <=6pt,shorten >=2pt] (O3c) to[bend right=60,looseness=2] (sb3c);
            %
            \draw [rex,semithick,shorten <=6pt,shorten >=2pt] (O4) to[out=45,in=90] (C5);
            \draw [rex,semithick,shorten <=2pt,shorten >=2pt] (db5) to[bend left=60,looseness=2] (O5a);
            \draw [rex,semithick,shorten <=6pt,shorten >=2pt] (O5b) to[out=180,in=-90,out looseness=2] (C4);
            \draw [rex,semithick,shorten <=2pt,shorten >=2pt] (db4) to[bend left=60,looseness=2] (O4);
            %
            \draw [rex,semithick,shorten <=6pt,shorten >=6pt] (S7) to[out=90,in=0,in looseness=2] (O6b);
            \draw [rex,semithick,shorten <=2pt,shorten >=6pt] (sb6) to[bend left=90,looseness=3] (O6a);
            %
            \draw [rex,semithick,shorten <=6pt,shorten >=2pt] (O8a) to[bend left=90,looseness=3] (sb8a);
            \draw [rex,semithick,shorten <=2pt,shorten >=2pt] (sb8b) to[bend right=60,looseness=1.5] (sb8c);
            \draw [rex,semithick,shorten <=2pt,shorten >=2pt] (sb8d) to[bend left=60,looseness=1.5] (sb8e);
        }
        \caption{Ozonolysis of alkenes.}
        \label{fig:Ozonolysis}
    \end{figure}
    \begin{itemize}
        \item The first step of the reaction is a concerted $3+2$ addition.
        \item The ozonide intermediate is more stable than the molozonide owing to its symmetry.
        \item Motivation for the last intermediate to split is eliminating charge separation.
    \end{itemize}
    \item Dihydroxylation of alkene (creation of a 1,2-diol).
    \item General form.
    \begin{equation*}
        \ce{R-= ->[1. OsO4][2. NaHSO3] R-(-OH)--OH}
    \end{equation*}
    \begin{itemize}
        \item Stereospecific (cis).
        \begin{itemize}
            \item Compare to bromination, which gives the trans-product (the difference is for mechanistic reasons).
        \end{itemize}
        \item The product is a 1,2-diol, or a vicinal diol.
    \end{itemize}
    \item Mechanism.
    \begin{figure}[H]
        \centering
        \schemestart
            \chemfig{R-[:-30]@{C1}=^[@{db1}6]}
            \arrow(.-30--.180){0}[,0.5]
            \chemfig{@{Os2}Os(=[1]O)(=[@{db2a}3]O)(=[@{db2b}5]@{O2}O)(=[7]O)}
            \arrow
            \chemname{\chemfig[atom sep=3em]{*5((<[:-80,0.7]H)(-[:-144,0.7]H)<:O-Os(=[1,0.7]O)(=[7,0.7]O)-O>:(<[:80,0.7]H)(-[,0.7]R)-)}}{Osmic ester}
            \arrow{->[\ce{NaHSO3}]}[,1.4]
            \chemfig{R-[:-30](-[:30]OH)-[6]-[:-30]OH}
        \schemestop
        \chemmove{
            \draw [rex,semithick,shorten <=3pt,shorten >=2pt] (db1) to[bend left=30,looseness=1.2] (O2);
            \draw [rex,semithick,shorten <=3pt,shorten >=2pt] (db2b) to[bend left=60,looseness=2] (Os2);
            \draw [rex,semithick,shorten <=3pt,shorten >=3pt] (db2a) to[out=-135,in=30] (C1);
        }
        \caption{Dihydroxylation.}
        \label{fig:dihydroxylation}
    \end{figure}
    \begin{itemize}
        \item The first step is concerted, once again.
        \item Osmium gets reduced in the first step (oxidation number goes from $+8$ to $+6$).
        \item In the second step, sodium bisulfite cleaves the two osmium oxygen bonds in a very complex process.
    \end{itemize}
    \item Problems: \ce{OsO4} is very expensive and very toxic.
    \item Solutions:
    \item UpJohn process (1976).
    \begin{itemize}
        \item The same as dihydroxylation but with only $1\%$ \ce{OsO4} and NMO (N-methylmorpholine oxide) and \ce{H2O} added second instead of \ce{NaHSO3}.
    \end{itemize}
    \item Sharpless asymmetric dihydroxylation.
    \begin{itemize}
        \item Nobel prize (2001).
        \item Gives high ee for each product.
        \item Conditions are catalytic potassium osmium salt (\ce{K2OsO2(OH)4}), potassium carbonate (\ce{K2CO3}), and potassium iron cyanate (\ce{K3Fe(CN)6}).
    \end{itemize}
    \item Alkene dihydrogenation.
\end{itemize}




\end{document}