\documentclass[../notes.tex]{subfiles}

\pagestyle{main}
\renewcommand{\chaptermark}[1]{\markboth{\chaptername\ \thechapter\ (#1)}{}}
\setcounter{chapter}{6}

\begin{document}




\chapter{Reactions of Alkynes}
\section{Methods of Hydration}
\begin{itemize}
    \item \marginnote{11/9:}Mukaiyama Hydration.
    \begin{itemize}
        \item A "greener" method.
    \end{itemize}
    \item General form.
    \begin{equation*}
        \ce{R-= ->[Co(acac)2, O2][PhSiH3] R-(-OH)-}
    \end{equation*}
    \item Mechanism:
    \begin{figure}[h!]
        \centering
        \footnotesize
        \begin{subfigure}[b]{\linewidth}
            \centering
            \chemfig{Co?(-[:35]O=[::-35](-[::60])-[::-60]=_[::-60](-[::60])-[::-60]O?)(-[:215]O=[::-35](-[::60])-[::-60]=_[::-60](-[::60])-[::-60]O?)}
            \caption{Cobalt catalyst.}
            \label{fig:mukaiyamaHydrationa}
        \end{subfigure}\\[2em]
        \begin{subfigure}[b]{\linewidth}
            \centering
            \schemestart
                \chemfig{R-[:30]=_[:-30]}
                \arrow{->[\footnotesize[\ce{Co-H}]]}[,1.3]
                \chemfig{R-[:30](-[2]{[Co]})-[:-30]-[:30]H}
                \arrow{->[\footnotesize\ce{O2}]}
                \chemfig{R-[:30](-[2]O-[:30]O-[2]{[Co]})-[:-30]-[:30]H}
                \arrow{->[\footnotesize\ce{PhSiH3}]}[,1.3]
                \chemfig{R-[:30](-[2]OH)-[:-30]-[:30]H}
            \schemestop
            \caption{Mechanism.}
            \label{fig:mukaiyamaHydrationb}
        \end{subfigure}
        \caption{Mukaiyama hydration mechanism.}
        \label{fig:mukaiyamaHydration}
    \end{figure}
    \begin{itemize}
        \item Note that \ce{Co(acac)2} is a catalyst.
    \end{itemize}
    \item Hydroboration/oxidation of alkene.
    \begin{itemize}
        \item Nobel Prize (1979).
        \item Discovered by a UChicago undergrad, H. C. Brown.
    \end{itemize}
    \item General form.
    \begin{equation*}
        \ce{R-= ->[1. BH3][2. H2O2, NaOH] R---OH}
    \end{equation*}
    \begin{itemize}
        \item Regioselective (anti-Markovnikov).
        \item Stereospecific (\ce{H} and \ce{OH} add cis).
        \item Covers \ce{BH3} with its empty $p$ orbital that makes it a good Lewis acid and dimerizes to \ce{B2H6} with its "3C-2e" bond in the gas phase.
    \end{itemize}
    \item Mechanism:
    \begin{figure}[h!]
        \centering
        \footnotesize
        \begin{subfigure}[b]{\linewidth}
            \centering
            \schemestart
                \chemfig{R-[:30]@{C1}=_[@{db}:-30]}
                \+
                \chemfig{H-[@{sb}]@{B2}B(<:[:20]H)(<[:-20]H)}
                \arrow
                \chemleft{[}
                    \chemfig{R-[:30]@{C3a}\phantom{}?-[:-30,1.3]@{C3b}-[:60,1.1,,,dashed]B(<[:-20]H)(<:[:20]H)-[:150,0.9,,,dashed]H?[,,dashed]}
                \chemright{]^\ddagger}
                \arrow
                \chemfig{R-[:30](-[2]H)-[:-30]-[:30]B(<:[:20]H)(<[:-20]H)}
                % \arrow{->[\footnotesize 2 \chemfig[atom sep=1.4em]{R-[:30]=_[:-30]}]}[,1.5]
                % \chemfig{@{R5}R-[:30]-[:-30]@{C5}-[:30]B}
            \schemestop
            \chemmove{
                \draw [rex,semithick,shorten <=4pt,shorten >=2pt] (db) to[out=-120,in=-90] (B2);
                \draw [rex,semithick,shorten <=2pt,shorten >=2pt] (sb) to[out=90,in=90] (C1);
                \draw [dashed,-] ([xshift=1pt,yshift=2pt]C3a.center) -- ([xshift=1pt,yshift=2pt]C3b.center);
                % \path (R5) node[left]{$\bigg($} -- (C5) node[right]{$\bigg)_3$};
            }
            \caption{Part I.}
            \label{fig:hydroborationa}
        \end{subfigure}\\[2em]
        \begin{subfigure}[b]{\linewidth}
            \centering
            \schemestart
                \chemfig{R-[:30]-[:-30]-[:30]@{B1}B-[:-30]@{C1}-[:30]-[:-30]@{R1}R}
                \+{1.5em}
                \chemfig{@{O2}\charge{90=\:,135:3pt=$\ominus$}{O}-O-H}
                \arrow
                \chemfig{R-[:30]-[:-30]-[@{sb1}:30]\charge{45:3pt=$\ominus$}{B}(-[2]@{O3a}O-[@{sb2}:30]@{O3b}O-[2]H)-[:-30]@{C3}-[:30]-[:-30]@{R3}R}
                \arrow{->[][-\ce{OH-}]}
                \chemfig{R-[:30]-[:-30]-[:30]O-[:-30]B-[:30]@{C4}-[:-30]-[:30]@{R4}R}
            \schemestop
            \chemmove{
                \path (C1) node[left]{$\bigg($} -- (R1) node[right]{$\bigg)_2$};
                \draw [rex,semithick,shorten <=6pt,shorten >=2pt] (O2) to[out=90,in=90] (B1);
                \path (C3) node[left]{$\bigg($} -- (R3) node[right]{$\bigg)_2$};
                \draw [rex,semithick,shorten <=2pt,shorten >=2pt] (sb1) to[bend left=30] (O3a);
                \draw [rex,semithick,shorten <=2pt,shorten >=2pt] (sb2) to[bend left=60,looseness=2] (O3b);
                \path (C4) node[left]{$\bigg($} -- (R4) node[right]{$\bigg)_2$};
            }
            \caption{Part II: Oxidative cleavage of the \ce{B-C} bond.}
            \label{fig:hydroborationb}
        \end{subfigure}\\[2em]
        \begin{subfigure}[b]{\linewidth}
            \centering
            \schemestart
                \chemfig{R{(}CH_2{)}_2O-[:30]@{B1}B-[:-30]{[}O{(}CH_2{)}_2R{]}_2}
                \arrow{<->>[\footnotesize\chemfig[atom sep=1.4em]{@{O2}\charge{90=\:,135:3pt=$\ominus$}{O}-H}]}[,1.2]
                \chemfig{R{(}CH_2{)}_2@{O3}O-[@{sb3}:30]\charge{45:3pt=$\ominus$}{B}(-[2]O-[:30]H)-[:-30]{[}O{(}CH_2{)}_2R{]}_2}
                \arrow{<=>}
                \chemfig{R{(}CH2{)}_2@{O4}\charge{90=\:,45:3pt=$\ominus$}{O}}
                \+
                \chemfig{@{H5}H-[@{sb5}]@{O5}O-B-{[}O{(}CH_2{)}_2R{]}_2}
                \arrow{->}[-90]
                \chemfig{R{(}CH2{)}_2OH}
                \+
                \chemfig{{(}\charge{45=$\oplus$}{Na}{)}\hspace{1em}\charge{135=$\ominus$}{O}-B-{[}O{(}CH_2{)}_2R{]}_2}
            \schemestop
            \chemmove{
                \draw [rex,semithick,shorten <=6pt,shorten >=2pt] (O2) to[bend right=90] (B1);
                \draw [rex,semithick,shorten <=2pt,shorten >=2pt] (sb3) to[bend right=60,looseness=2] (O3);
                \draw [rex,semithick,shorten <=6pt,shorten >=2pt] (O4) to[bend left=90,looseness=2] (H5);
                \draw [rex,semithick,shorten <=2pt,shorten >=2pt] (sb5) to[bend left=90,looseness=3] (O5);
            }
            \caption{Part III: Hydrolysis of boric acid ester.}
            \label{fig:hydroborationc}
        \end{subfigure}
        \caption{Hydroboration mechanism.}
        \label{fig:hydroboration}
    \end{figure}
    \begin{itemize}
        \item The step in Part I is a concerted step (this is key for cis addition).
        \item \ce{BH3} adds the way it does in part I (with the \ce{H} put on the more substituted carbon) due to sterics (bulky group attaches to the less bulky side) and electronics (the $\pi$-bonding electrons of the alkene connecting to the boron first will create a partial positive charge, and it is more stable to have that partial positive on the more substituted position).
        \item The alkene does not attack an \ce{H} on \ce{BH3} because said \ce{H}'s are not acidic (it is the boron that is electron deficient).
        \item Each part repeats an additional two times with the product of the $n^\text{th}$ run the reactant of the $(n+1)^\text{th}$ run to create the reactant of the next part.
        \item In part II, mixing \ce{H2O2} with \ce{OH-} yields a deprotonated peroxide (\ce{O2H-}).
        \item The final product of part II is the boric acid ester.
        \item Part III has a borate salt as a final byproduct.
        \item The mechanism implies that the hydrogen added comes from \ce{BH3}, and the oxygen added (as part of the hydroxide) comes from \ce{H2O2}.
    \end{itemize}
    \item If we add to an alkene borane, and then a strong acid and heat, we end up hydrogenating it.
    \item Summary of alcohol synthesis:
    \begin{itemize}
        \item If you want Markovnikov addition, use\dots
        \begin{itemize}
            \item Acid-catalyzed hydration.
            \item Oxymercuration/reduction.
            \item Mukaiyama hydration.
        \end{itemize}
        \item If you want anti-Markovnikov addition, use\dots
        \begin{itemize}
            \item Hydroboration.
        \end{itemize}
    \end{itemize}
    \item Ozonolysis of alkenes.
    \begin{itemize}
        \item An \textbf{oxidation} reaction (adding oxygen or removing hydrogen).
        \item Treat an alkene with ozone and dimethyl sulfide to cleave the \ce{C=C} bond and add oxygen onto each carbon, forming carbonyls (and associated groups).
        \item Note that if we consider the resonance structures of ozone, we will find that charger separation is unavoidable, i.e., that the molecule must have a plus and a minus charge somewhere. This makes it very reactive.
    \end{itemize}
    \item Mechanism.
    \begin{figure}[h!]
        \centering
        \footnotesize
        \schemestart
            \chemfig{R>:[:70]@{C1}(<[:-70]H)=_[@{db1}2](<[:70]H)<:[:110]R}
            \arrow{0}[,0.5]
            \chemfig{@{O2a}\charge{0=\:,-90=\:,180=\:}{O}-[:60]@{O2b}\charge{0=\:}{O}=[@{db2}:120]@{O2c}\charge{[extra sep=1.5pt]45=\:,135=\:}{O}}
            \arrow
            \chemname{\chemfig[atom sep=3em]{*5((<:[:-150,0.7]R)(<[:-110,0.7]H)-[@{sb3a}]\charge{[extra sep=1.5pt]-45=\:,-135=\:}{O}-[@{sb3b}]@{O3b}\charge{[extra sep=1.5pt]45=\:,-45=\:}{O}-@{O3c}\charge{[extra sep=1.5pt]45=\:,135=\:}{O}-[@{sb3c}](<:[:150,0.7]R)(<[:110,0.7]H)-[@{sb3d}])}}{Molozonide}
            \arrow
            \chemfig{R-[:30]@{C4}(=[@{db4}2]@{O4}\charge{[extra sep=1.5pt]45=\:,135=\:}{O})-[:-30]H}
            \arrow{0}[,0]\+{,,-2em}
            \chemfig{R-[:-30]@{C5}(=[@{db5}6]@{O5a}\charge{[extra sep=1.5pt]-45=\:}{O}-[:-150]@{O5b}\charge{90=\:,180=\:,-90=\:}{O})-[:30]H}
            \arrow
            \chemname{\chemfig[atom sep=3em]{[:18]*5(@{O6a}\charge{-90=\:,180=\:}{O}?-[@{sb6}]@{O6b}\charge{0=\:,-90=\:}{O}-(<:[:38,0.7]R)(<[:-2,0.7]H)-\charge{[extra sep=1.5pt]45=\:,135=\:}{O}-?(<[:182,0.7]H)(<:[:142,0.7]R))}}{Ozonide}
            \arrow{->[*{0}{\chemfig[atom sep=1.4em]{-[:30]@{S7}\charge{90=\:,-90=\:}{S}-[:-30]}}]}[-90]
            \chemfig{@{O8a}\charge{0=\:,-90=\:,180=\:,45:3pt=$\ominus$}{O}-[@{sb8a}2](<[:150]H)(<:[:110]R)-[@{sb8b}:30]\charge{[extra sep=1.5pt]45=\:,135=\:}{O}-[@{sb8c}:-30](<[:30]H)(<:[:70]R)-[@{sb8d}6]\charge{180=\:,-90=\:}{O}-[@{sb8e}:-30]\charge{90=\:,45:3pt=$\oplus$}{S}(-[6])-[:30]}
            \arrow[180]
            \subscheme{
                \chemfig{H-[:-30](=[6]\charge{[extra sep=1.5pt]-45=\:,-135=\:}{O})-[:30]R}
                \arrow{0}[,0]\+{,,2em}
                \chemfig{H-[2](=[:150]\charge{90=\:,180=\:}{O})-[:30]R}
                \arrow(.east--.160){0}[,0]\+{,,2em}
                \chemname[-1em]{\chemfig{-[:30]\charge{-90=\:}{S}(=[2]\charge{[extra sep=1.5pt]45=\:,135=\:}{O})-[:-30]}}{DMSO}
            }
        \schemestop
        \chemmove{
            \draw [rex,semithick,shorten <=4pt,shorten >=2pt] (db1) to[bend right=30] (O2c);
            \draw [rex,semithick,shorten <=3pt,shorten >=2pt] (db2) to[bend left=60,looseness=2.5] (O2b);
            \draw [rex,semithick,shorten <=5pt,shorten >=4pt] (O2a) to[bend left=15] (C1);
            %
            \draw [rex,semithick,shorten <=2pt,shorten >=2pt] (sb3d) to[bend left=30,looseness=1.2] (sb3a);
            \draw [rex,semithick,shorten <=2pt,shorten >=6pt] (sb3b) to[bend right=90,looseness=3] (O3b);
            \draw [rex,semithick,shorten <=6pt,shorten >=2pt] (O3c) to[bend right=60,looseness=2] (sb3c);
            %
            \draw [rex,semithick,shorten <=6pt,shorten >=2pt] (O4) to[out=45,in=90] (C5);
            \draw [rex,semithick,shorten <=2pt,shorten >=2pt] (db5) to[bend left=60,looseness=2] (O5a);
            \draw [rex,semithick,shorten <=6pt,shorten >=2pt] (O5b) to[out=180,in=-90,out looseness=2] (C4);
            \draw [rex,semithick,shorten <=2pt,shorten >=2pt] (db4) to[bend left=60,looseness=2] (O4);
            %
            \draw [rex,semithick,shorten <=6pt,shorten >=6pt] (S7) to[out=90,in=0,in looseness=2] (O6b);
            \draw [rex,semithick,shorten <=2pt,shorten >=6pt] (sb6) to[bend left=90,looseness=3] (O6a);
            %
            \draw [rex,semithick,shorten <=6pt,shorten >=2pt] (O8a) to[bend left=90,looseness=3] (sb8a);
            \draw [rex,semithick,shorten <=2pt,shorten >=2pt] (sb8b) to[bend right=60,looseness=1.5] (sb8c);
            \draw [rex,semithick,shorten <=2pt,shorten >=2pt] (sb8d) to[bend left=60,looseness=1.5] (sb8e);
        }
        \caption{Ozonolysis of alkenes.}
        \label{fig:Ozonolysis}
    \end{figure}
    \begin{itemize}
        \item The first step of the reaction is a concerted $3+2$ addition.
        \item The ozonide intermediate is more stable than the molozonide owing to its symmetry.
        \item Motivation for the last intermediate to split is eliminating charge separation.
    \end{itemize}
    \item Dihydroxylation of alkene (creation of a 1,2-diol).
    \item General form.
    \begin{equation*}
        \ce{R-= ->[1. OsO4][2. NaHSO3] R-(-OH)--OH}
    \end{equation*}
    \begin{itemize}
        \item Stereospecific (cis).
        \begin{itemize}
            \item Compare to bromination, which gives the trans-product (the difference is for mechanistic reasons).
        \end{itemize}
        \item The product is a 1,2-diol, or a vicinal diol.
    \end{itemize}
    \item Mechanism.
    \begin{figure}[H]
        \centering
        \footnotesize
        \schemestart
            \chemfig{R-[:-30]@{C1}=^[@{db1}6]}
            \arrow(.-30--.180){0}[,0.5]
            \chemfig{@{Os2}Os(=[1]O)(=[@{db2a}3]O)(=[@{db2b}5]@{O2}O)(=[7]O)}
            \arrow
            \chemname{\chemfig[atom sep=3em]{*5((<[:-80,0.7]H)(-[:-144,0.7]H)<:O-Os(=[1,0.7]O)(=[7,0.7]O)-O>:(<[:80,0.7]H)(-[,0.7]R)-)}}{Osmic ester}
            \arrow{->[\ce{NaHSO3}]}[,1.4]
            \chemfig{R-[:-30](-[:30]OH)-[6]-[:-30]OH}
        \schemestop
        \chemmove{
            \draw [rex,semithick,shorten <=3pt,shorten >=2pt] (db1) to[bend left=30,looseness=1.2] (O2);
            \draw [rex,semithick,shorten <=3pt,shorten >=2pt] (db2b) to[bend left=60,looseness=2] (Os2);
            \draw [rex,semithick,shorten <=3pt,shorten >=3pt] (db2a) to[out=-135,in=30] (C1);
        }
        \caption{Dihydroxylation.}
        \label{fig:dihydroxylation}
    \end{figure}
    \begin{itemize}
        \item The first step is concerted, once again.
        \item Osmium gets reduced in the first step (oxidation number goes from $+8$ to $+6$).
        \item In the second step, sodium bisulfite cleaves the two osmium oxygen bonds in a very complex process.
    \end{itemize}
    \item Problems: \ce{OsO4} is very expensive and very toxic.
    \item Solutions:
    \item UpJohn process (1976).
    \begin{itemize}
        \item The same as dihydroxylation but with only $1\%$ \ce{OsO4} and NMO (N-methylmorpholine oxide) and \ce{H2O} added second instead of \ce{NaHSO3}.
    \end{itemize}
    \item Sharpless asymmetric dihydroxylation.
    \begin{itemize}
        \item Nobel prize (2001).
        \item Gives high ee for each product.
        \item Conditions are catalytic potassium osmium salt (\ce{K2OsO2(OH)4}), potassium carbonate (\ce{K2CO3}), and potassium iron cyanate (\ce{K3Fe(CN)6}).
    \end{itemize}
    \item Alkene dihydrogenation.
    \item \marginnote{11/13:}General form:
    \begin{equation*}
        \ce{R-= ->[H2][cat. Pt] R--}
    \end{equation*}
    \begin{itemize}
        \item A reduction reaction.
        \item The catalyst can be \ce{Pd}, \ce{Ru}, \ce{Rh}, \ce{Ir}, etc.
        \item This reaction happens on the solid surface of a metal catalyst.
        \item Most of the time, this is cis-addition (as we can determine with deuterium labeling).
    \end{itemize}
\end{itemize}



\section{Alkynes}
\begin{itemize}
    \item Review of bonding.
    \begin{itemize}
        \item Consider acetylene, or ethyne (\ce{H-C#C-H}).
        \begin{itemize}
            \item Bond angle $\ang{180}$.
            \item Linear
            \item $sp$ hybridization.
            \item Orbital diagram (similar to homework 1.7b).
        \end{itemize}
        \item Driving force: Break weaker $\pi$ bonds.
    \end{itemize}
    \item IUPAC nomenclature.
    \begin{itemize}
        \item If there is a stereocenter, we need (R/S). If there is cis/trans, we need that, too.
        \item Same rules as for alkenes except with "-yne."
        \item No Z/E for alkyne.
        \item Alkenes have higher priority than alkynes, e.g., we have but-1-ene-3-yne, not but-1-yne-3-ene.
        \item Alkenes have higher priority than alkynes, have higher priority than halogens, e.g., we have 3-bromo-3-methyl-1-butyne.
    \end{itemize}
    \item Acidity of terminal alkynes.
    \begin{itemize}
        \item Recall that $sp$ hydrogens are more acidic than $sp^2$ hydrogens, are more acidic than $sp^3$ hydrogens (more $s$ character means that the charge on the conjugate base is held closer to the positive nucleus and thus stabilized better).
        \item Indeed, \ce{R-C#C-H} is a reasonable Br\o nsted acid (it can react with a strong base).
        \begin{itemize}
            \item For example, acetylene and sodium amide react to establish an acid-base equilibrium to the right.
        \end{itemize}
        \item Take home message: Strong bases can remove hydrogen from terminal alkynes to give \ce{R-C#C^-}.
        \item Two more strong bases (that can fully remove a hydrogen from a terminal alkyne): \ce{NaH} (sodium hydride) and LDA (lithium diisopropylamide).
        \item \ce{NaOH} cannot remove a hydrogen from a terminal alkyne.
    \end{itemize}
    \item Reactions of alkynes.
    \begin{itemize}
        \item Tip: Learn alkyne reactions simply by making an analogy to an alkene reaction.
    \end{itemize}
    \item Hydrohalogenation.
    \vspace{2em}
    \begin{figure}[h!]
        \centering
        \footnotesize
        \begin{subfigure}[b]{\linewidth}
            \centering
            \schemestart
                \chemfig{R-~[@{CC}]-H}
                \+
                \chemfig{@{H}H-[@{sb}]@{Cl}\charge{-90=\:,0=\:,90=\:}{Cl}}
                \arrow{<=>}
                \chemfig{R-[:30]@{C+}\charge{135:3pt=$\oplus$}{}=_[:-30](-[:30]H)(-[6]H)}
                \+
                \chemfig{@{Cl2}\charge{0=\:,90=\:,180=\:,270=\:,45:3pt=$\ominus$}{Cl}}
                \arrow
                \chemfig{R-[:30](-[2]Cl)=_[:-30](-[:30]H)}
            \schemestop
            \chemmove{
                \draw [rex,semithick,shorten <=5pt,shorten >=3pt] (CC) to[bend left=90,looseness=1.5] (H);
                \draw [rex,semithick,shorten <=3pt,shorten >=6pt] (sb) to[bend left=90,looseness=3] (Cl);
                \draw [rex,semithick,shorten <=6pt,shorten >=3pt] (Cl2) to[bend right=90] (C+);
            }
            \caption{One equivalent \ce{HBr}.}
            \label{fig:hydrohalogenationAlkynea}
        \end{subfigure}\\[2em]
        \begin{subfigure}[b]{\linewidth}
            \centering
            \schemestart
                \chemfig{R-[:30](-[2]Cl)=^[@{CC}:-30]}
                \arrow{0}[,0]\+{1em,1em}
                \chemfig{@{H}H-[@{sb}]@{Cl}\charge{-90=\:,0=\:,90=\:}{Cl}}
                \arrow{<=>}
                \chemfig{R-[:30]@{C+}\charge{135:3pt=$\oplus$}{}(-[2]Cl)-[:-30](-[:30]H)}
                \arrow{0}[,0]\+{1em,1em}
                \chemfig{@{Cl2}\charge{0=\:,90=\:,180=\:,270=\:,45:3pt=$\ominus$}{Cl}}
                \arrow
                \chemfig{R-[:30](-[:70]Cl)(-[:110]Cl)-[:-30](-[:30]H)}
            \schemestop
            \chemmove{
                \draw [rex,semithick,shorten <=4pt,shorten >=3pt] (CC) to[bend left=60,looseness=1.2] (H);
                \draw [rex,semithick,shorten <=3pt,shorten >=6pt] (sb) to[bend left=90,looseness=3] (Cl);
                \draw [rex,semithick,shorten <=6pt,shorten >=3pt] (Cl2) to[bend right=50] (C+);
            }
            \caption{Another equivalent \ce{HBr}.}
            \label{fig:hydrohalogenationAlkyneb}
        \end{subfigure}
        \caption{Hydrohalogenation mechanism (alkynes).}
        \label{fig:hydrohalogenationAlkyne}
    \end{figure}
    \begin{itemize}
        \item Two equivalents of \ce{HBr} yields a \textbf{geminal dichloride}.
        \item Still Markovnikov addition.
        \item If we wanted to form a viscinal (or 1,2-) dichloride, we would use chlorination, but if we want to form the geminal chloride, we must start with an alkyne.
    \end{itemize}
    \item Halogenation.
    \begin{itemize}
        \item Similarly, one equivalent yields a trans alkene.
        \item Two equivalents yield a tetrahalo alkyne.
    \end{itemize}
    \item Acid-catalyzed hydration.
    \begin{figure}[H]
        \centering
        \footnotesize
        \schemestart
            \chemfig{R-~-H}
            \arrow{->[\ce{H2SO4}, \ce{H2O}][cat. \ce{HgSO4}]}[,2]
            \chemfig{R-[:30](-[2]OH)=_[:-30](-[6]H)-[:30]H}
            \arrow{<->>}
            \chemfig{R-[:30](=[2]O)-[:-30](-[6]H)(<[:-10]H)(<:[:30]H)}
        \schemestop
        \caption{Hydration mechanism (alkynes).}
        \label{fig:hydrationAlkyne}
    \end{figure}
    \begin{itemize}
        \item For an alkyne, we need a more forcing condition. In particular, we will add catalytic \ce{HgSO4}.
        \item After running once, we will form an enol.
        \begin{itemize}
            \item Enols are unstable and undergo enol-keto tautomerizations, forming a ketone.
            \item If we are asked to draw the products of this reaction, draw \emph{only} the ketone.
            \item The tautomerization favors the ketone for thermodynamic reasons: The ketone is more stable (by about $\SI{15}{\kilo\calorie\per\mole}$), and the \ce{O-H} and \ce{C-H} bonds have similar BDEs.
        \end{itemize}
        \item This is Markovnikov.
        \item A good method for ketone synthesis: Alkyne to ketone.
        \item We do not need to know the mechanism because the introduction of the mercury catalyst goes beyond this class.
        \item Know, however, that alkyne hydration requires a more forcing condition because alkynes' hybridization leads to tighter holding of electrons relative to alkenes. Thus, we say that alkenes are more electron rich.
        \item There are some alternative greener methods, but we will not cover them.
    \end{itemize}
    \item Hydroboration.
    \begin{figure}[h!]
        \centering
        \footnotesize
        \begin{subfigure}[b]{\linewidth}
            \centering
                \schemestart
                \chemfig{R-~-R'}
                \arrow{->[\ce{BH3}]}
                \chemfig{@{R}R-[:30](-[2]H)=_[:-30]@{C}(-[6]R')-[:30]B}
                \arrow{->[\ce{NaOH}][\ce{H2O2}]}[,1.2]
                \chemfig{R-[:30](-[2]H)=_[:-30](-[6]R')-[:30]OH}
                \arrow{<->>}
                \chemfig{R-[:30](-[:70]H)(-[:110]H)-[:-30](-[6]R')=^[:30]O}
            \schemestop
            \chemmove{
                \path (R) node[left,yshift=1mm]{$\left(\rule{0cm}{1.2cm}\right.$} -- (C) node[right,yshift=1mm]{$\left.\rule{0cm}{1.2cm}\right)_3$};
            }
            \vspace{1em}
            \caption{Alkyne hydroboration.}
            \label{fig:hydroborationAlkynea}
        \end{subfigure}\\[1em]
        \begin{subfigure}[b]{\linewidth}
            \centering
                \schemestart
                \chemfig{R-~}
                \arrow{->[\ce{BH3}]}
                \chemfig{@{R}R-[:30](-[2]H)=_[:-30]@{C}-[:30]B}
                \arrow{->[\ce{BH3}]}
                \chemfig{R-[:30]-[:-30](-[:30]B(-[::60]!{wave})(-[::-60]!{wave}))(-[6]B(-[::60]!{wave})(-[::-60]!{wave}))}
            \schemestop
            \chemmove{
                \path (R) node[left,yshift=5mm]{$\left(\rule{0cm}{8mm}\right.$} -- (C) node[right,yshift=5mm]{$\left.\rule{0cm}{8mm}\right)_3$};
            }
            \caption{Over hydroboration.}
            \label{fig:hydroborationAlkyneb}
        \end{subfigure}\\[1em]
        \begin{subfigure}[b]{\linewidth}
            \centering
                \schemestart
                \chemfig{R-~}
                \arrow{->[\ce{(sia)2BH}]}[,1.4]
                \chemfig{R-[:30](-[2]H)=_[:-30]-[:30]B{(sia)}_2}
                \arrow{->[\ce{NaOH}][\ce{H2O2}]}[,1.2]
                \chemfig{R-[:30](-[2]H)=_[:-30]-[:30]OH}
                \arrow
                \chemfig{R-[:30](-[:70]H)(-[:110]H)-[:-30]=^[:30]O}
            \schemestop
            \caption{Solving over hydroboration.}
            \label{fig:hydroborationAlkynec}
        \end{subfigure}
        \caption{Hydroboration mechanism (alkynes).}
        \label{fig:hydroborationAlkyne}
    \end{figure}
    \begin{itemize}
        \item Three equivalent of the reactant go through at once to form three equivalents of the product.
        \item The product results from typical hydroboration cis-addition followed by the keto-enol tautomerization.
        \item The \ce{R$'$} group in the normal hydroboration prevents boron from adding to the alkene again via steric hindrance.
        \item We can solve over hydroboration by using \ce{(sia)2B-H} instead of \ce{BH3}, which only works one molecule at a time and is too bulky for over hydroboration.
        \begin{itemize}
            \item The sia ligand is sec-isoamyl (5 carbons, prong at the end, bonds through the second carbon along the tail).
            \item The full name of \ce{(sia)2BH} is di-sec-iso-amylborane.
        \end{itemize}
    \end{itemize}
    \item Three ways to make ketones:
    \begin{enumerate}
        \item Ozonolysis of alkenes.
        \item Acid-catalyzed hydration of alkynes.
        \item Hydroboration of alkynes.
    \end{enumerate}
    \item Reduction (hydrogenation).
    \begin{itemize}
        \item The reaction is hard to stop at the alkene if we use catalytic platinum and hydrogen.
        \item To stop at the alkene stage, we can use a Lindlar catalyst (has some \ce{Pd}, \ce{CaCO3}, and \ce{PbO}).
        \begin{itemize}
            \item A \textbf{poisoned catalyst} that does not have the same reactivity as platinum. It can bind with the alkyne, but not the alkene.
        \end{itemize}
        \item Alternatively, we can use \ce{Ni2Br}.
        \item We can get the trans product with a special condition called dissolving metal reduction.
        \begin{equation*}
            \ce{R-#-R$'$ ->[2 Na][2 NH3] \emph{trans-}R-=-R$'$ + 2NaNH2}
        \end{equation*}
        \begin{figure}[h!]
            \centering
            \footnotesize
            \schemestart
                \chemfig{R-@{C1a}~[@{db1}]@{C1b}-R'}
                \arrow{->[\chemfig{@{Na2}\charge{180=\.}{Na}}]}
                \chemfig{\charge{45:3pt=$\oplus$}{Na}}
                \arrow{0}[,0]\+{1em,,-2.8em}
                \chemleft{[}\subscheme{
                    \chemfig{R-\charge{90=\.}{C}=@{C3}\charge{90=\:,45:3pt=$\ominus$}{C}-R'}
                    \arrow{<->}[-90]
                    \chemfig{R-\charge{90=\:,135:3pt=$\ominus$}{C}=\charge{90=\.}{C}-R'}
                }\chemright{]}
                \arrow{->[\chemfig{@{H5}H-[@{sb5}]@{N5}\charge{90=\:}{N}H_2}][-\ce{NaNH2}]}[,1.6]
                \chemfig{R-@{C6}\charge{90=\.}{C}=C(-[:60]H)(-[:-60]R')}
                \arrow{->[*{0}\chemfig{@{Na7}\charge{90=\.}{Na}}]}[-90]
                \subscheme{
                    \chemfig{\charge{45:3pt=$\oplus$}{Na}}
                    \arrow{0}[,0]\+{1em,,-2.4em}
                    \chemfig{R-[:-60]@{C9}\charge{[extra sep=1.5pt]-135=\:,45:3pt=$\ominus$}{C}=C(-[:60]H)(-[:-60]R')}
                }
                \arrow{->[*{0.-90}\chemfig{@{H10}H-[@{sb10}]@{N10}\charge{90=\:}{N}H_2}][*{0.90}-\ce{NaNH2}]}[180,1.6]
                \chemfig{R-[:-60](-[:-120]H)=(-[:60]H)(-[:-60]R')}
            \schemestop
            \chemmove{
                \draw [rex,semithick,shorten <=6pt,shorten >=4pt,arrows={-Stealth[harpoon]}] (Na2) to[out=180,in=80,in looseness=2] (C1b);
                \draw [rex,semithick,shorten <=4pt,shorten >=4pt,arrows={-Stealth[harpoon,swap]}] (db1) to[bend left=80,looseness=5] (C1b);
                \draw [rex,semithick,shorten <=4pt,shorten >=4pt,arrows={-Stealth[harpoon,swap]}] (db1) to[bend right=80,looseness=5] (C1a);
                %
                \draw [rex,semithick,shorten <=6pt,shorten >=2pt] (C3) to[out=90,in=90,looseness=1.5] (H5);
                \draw [rex,semithick,shorten <=2pt,shorten >=6pt] (sb5) to[bend left=90,looseness=4] (N5);
                %
                \draw [rex,semithick,shorten <=6pt,shorten >=2pt,arrows={-Stealth[harpoon]}] (Na7) to[out=90,in=-90] (C6);
                %
                \draw [rex,semithick,shorten <=6pt,shorten >=2pt] (C9) to[out=-135,in=-60,out looseness=2] ++(-0.6,0) to[out=120,in=90,out looseness=1.2] (H10);
                \draw [rex,semithick,shorten <=2pt,shorten >=6pt] (sb10) to[bend left=90,looseness=4] (N10);
            }
            \caption{Monohydrogenation of an alkyne.}
            \label{fig:hydrogenationAlkyne}
        \end{figure}
        \begin{itemize}
            \item Dissolve two equivalents of sodium in \ce{NH3}.
            \item Stereospecific (trans).
            \item Sodium is very electropositive, a single electron donor. On the other hand, the alkyne is electron poor.
            \item We favor the trans intermediate for steric reasons.
        \end{itemize}
    \end{itemize}
\end{itemize}




\end{document}