\documentclass[../notes.tex]{subfiles}

\pagestyle{main}
\renewcommand{\chaptermark}[1]{\markboth{\chaptername\ \thechapter\ (#1)}{}}
\setcounter{chapter}{3}

\begin{document}




\chapter{Stereochemistry}
\section{Intro to Chirality and Chiral Compounds}
\begin{itemize}
    \item \marginnote{10/26:}Today:
    \begin{enumerate}
        \item Stereoisomers / chirality center.
        \item Chirality test.
        \item Keeping track of stereoisomers (R/S system).
        \item Physical properties of enantiomers.
        \item Molecules with multiple chirality centers.
        \item Fischer projections.
        \item Meso compounds.
        \item Chiral molecules with no chirality center.
    \end{enumerate}
    \item \textbf{Achiral} (object): An object such that it and its mirror image are identical.
    \item \textbf{Chiral} (object): An object such that it and its mirror are nonidentical (cannot be superimposed).
    \item Single enantiomer drugs is about a \$100 billion industry.
    \begin{itemize}
        \item Biological molecules are chiral.
    \end{itemize}
    \item \textbf{Stereoisomers}: Same connectivity; different spatial arrangement of groups.
    \item \textbf{Enantiomers}: Non-super-imposable mirror images.
    \begin{itemize}
        \item E.g., 2-butanol.
    \end{itemize}
    \item \textbf{Diastereomers}: Stereoisomers that are not mirror images of each other.
    \begin{itemize}
        \item E.g., \emph{cis}- and \emph{trans}-2-butene.
    \end{itemize}
    \item \textbf{Chirality center}: A tetrahedral carbon that is bound to four different groups.
    \item Molecules with one chirality center are chiral and exist as a pair of enantiomers.
    \item Chirality test: Check for a \textbf{plane of symmetry}.
    \item \textbf{Plane of symmetry}: An imaginary plane that bisects the molecule such that the two halves are mirror images of each other.
    \item Lowest priority group away from you; clockwise 1,2,3 is R; counterclockwise is S.
    \item Enantiomers have the same boiling and melting point.
    \begin{itemize}
        \item They are only different when interacting with other chiral substances.
        \item They also rotate plane-polarized light different directions.
    \end{itemize}
    \item \textbf{Racemic mixture}: An equimolar mixture of enantiomers.
    \item \textbf{Enantiomeric excess}: The following quantity. \emph{Denoted by} \textbf{ee}. \emph{Given by}
    \begin{equation*}
        \text{ee} = \frac{(\text{moles enantiomer 1})-(\text{moles enantiomer 2})}{\text{total moles of both}}
    \end{equation*}
    \begin{itemize}
        \item $\text{ee}=0$ for a racemic mixture; $\text{ee}=100$ for an enantiomerically pure mixture.
    \end{itemize}
    \item How many possible stereoisomers?
    \begin{itemize}
        \item $2^n$ possible ones, where $n$ is the number of chirality centers.
    \end{itemize}
\end{itemize}




\end{document}