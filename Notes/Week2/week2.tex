\documentclass[../notes.tex]{subfiles}

\pagestyle{main}
\renewcommand{\chaptermark}[1]{\markboth{\chaptername\ \thechapter\ (#1)}{}}
\setcounter{chapter}{1}

\begin{document}




\chapter{Families of Carbon Compounds / Acids and Bases}
\section{Families of Carbon Compounds}
\begin{itemize}
    \item \marginnote{10/7:}Hydrocarbons:
    \begin{itemize}
        \item Alkanes (\ce{C_$n$H_{$2n+2$}}) and cycloalkanes \ce{C_$n$H_{2$n$}}.
        \item Alkenes (\ce{C_$n$H_{$2n$}}).
        \item Alkynes (\ce{C_$n$H_{$2n-2$}}).
    \end{itemize}
    \item Aromatic:
    \begin{itemize}
        \item Contains a benzene ring.
        \item All bonds $\sim\SI{140}{\angstrom}$.
        \item All carbons $sp^2$.
        \item Planar.
        \item $\pi$ electrons above and below the ring.
        \item Special stabilization.
    \end{itemize}
    \item Covers drawing dipoles.
    \item Polar and nonpolar molecules:
    \begin{itemize}
        \item $\text{Dipole}=\text{distance}\times\text{change between charges}$.
        \item $\mu=r\times Q$
        \item $\SI{1}{\debye}=\SI{3.336e-30}{\coulomb\meter}$.
        \item Analyzes molecules by drawing a Lewis structure, drawing a dipole along each bond, and drawing and labeling a net dipole, if applicable.
        \item Goes through a number of examples.
        \item Acetonitrile is a strong polar solvent.
    \end{itemize}
    \item \textbf{Functional group}: A common arrangement that determines shape, bonding physical and reactivity of organic compounds.
    \item Families of carbon compounds:
    \begin{itemize}
        \item Hydrocarbons: Aliphatic, aromatic.
        \item Methyl, ethyl, propyl, $R=\text{alkyl}$ groups.
        \item Phenyl: \ce{Ph{-}} or \ce{$\phi${-}}.
        \item Benzyl: \ce{Ph-CH2{-}}, \ce{C6H5CH2{-}}, \ce{Bn{-}}
        \item Compounds with \ce{R-Z} where Z is a heteroatom.
        \begin{itemize}
            \item If Z is a halogen X, then the halogroup makes it an alkyl halide or haloalkane.
        \end{itemize}
        \item Alkenyl halide: \ce{X-=}.
        \item Aryl halide: \ce{Ph-X}.
        \item Alcohols or phenols: \ce{R-OH}.
        \item Ether: \ce{R-O-R$'$}.
        \item Amines: \ce{NH2R}, \ce{NHRR$'$}, \ce{NRR$'$R$''$}.
        \item Thiols or mercathols: \ce{R-SH}.
        \item Carbonyl group: \ce{R-CO-R$'$}.
        \item Aldehyde: \ce{R-COH}.
        \item Ketone: \ce{R-CO-R$'$}.
        \item Carboxylic acid derivatives:
        \begin{itemize}
            \item Acid: \ce{R-COOH}.
            \item Ester: \ce{R-COOR$'$}.
            \item Acid chloride: \ce{R-COCl}.
            \item Acid halide: \ce{R-COX}.
            \item Amide: \ce{R-CONH2}.
            \item Acid anhydride: \ce{R-COOCO-R$'$}.
        \end{itemize}
        \item Nitrile: \ce{R-C#N}.
        \item Acrylonitrile: \ce{=-C#N}.
    \end{itemize}
\end{itemize}



\section{Discussion Section}
\begin{itemize}
    \item ACS in-text citations should be in superscripts as a list of number with no brackets or parentheses.
    \item Molecular formulas are \ce{C2H6O}, not \ce{C2H5OH} or \ce{CH3CH2OH}.
    \item Make a table if you have a lot of data to put in (make it readable!).
    \item Distillation:
    \begin{itemize}
        \item We need a boiling chip and stir bar inside the flask.
        \item Vapor comes up from a round-bottomed flask, encounters a rubber stopper and gets diverted through a condenser instead.
        \item Make use of countercurrent exchange and increase pressure by inflowing water in the gravitationally lower portion of the condenser.
        \item Boiling chip is a coarse material with a lof of micropores inside.
        \item The surface energy is reduced when the fluid is inside the micropores; within, it can more easily become a gas.
    \end{itemize}
    \item As the mole fraction $\chi$ of a substance \ce{A} increases\dots
    \item Raoult's law:
    \begin{equation*}
        P_\text{total} = \frac{P_A\chi_A}{P_B\chi_B} = \frac{P_A\chi_A}{P_B(1-\chi_A)}
    \end{equation*}
    \item Dalton's law: The total pressure is equal to the sum of the partial pressures.
\end{itemize}



\section{Intermolecular Forces and IR Spectroscopy}
\begin{itemize}
    \item \marginnote{10/12:}Intermolecular forces and physical properties.
    \item \textbf{Boiling point}: The temperature at which the vapor pressure is equal to the pressure of the atmosphere above.
    \begin{itemize}
        \item The stronger the intermolecular forces, the higher the boiling point.
        \item The higher the molecular weight, the higher the boiling point.
    \end{itemize}
    \item \textbf{Melting point}: The temperature at which the crystalline solid and liquid are in equilibrium.
    \begin{itemize}
        \item The stronger the intermolecular forces, the higher the melting point.
        \item The more symmetrical, the higher the melting point.
    \end{itemize}
    \item Solubility.
    \item Intermolecular forces.
    \begin{itemize}
        \item All electrostatic attractions related to bond polarity.
        \item 3 types: Dipole-dipole forces, hydrogen bonding, and dispersion forces.
    \end{itemize}
    \item \textbf{Dipoloe-dipole forces}: Attraction between opposite poles ($\SIrange[per-mode=symbol]{1}{3}{\kilo\calorie\per\mole}$).
    \item \textbf{Hydrogen bonding}: Dipole-dipole interaction between \ce{H}-atoms bonded to \ce{O}, \ce{N}, \ce{F} ($\SIrange[per-mode=symbol]{2}{10}{\kilo\calorie\per\mole}$).
    \item \textbf{Dispersion forces}: Weak ($<\SI[per-mode=symbol]{1}{\kilo\calorie\per\mole}$). Momentary distortion of the electron cloud (temporary dipole). Induces dipoles in surrounding molecules. \emph{Also known as} \textbf{London forces}.
    \begin{itemize}
        \item Depends on \textbf{relative polarizability}.
        \item Dependes on the surface area of the molecule --- more surface area means more distance electrons can spread apart.
    \end{itemize}
    \item \textbf{Relative polarizability}: How far valence electrons are from the nucleus.
    \item Solubility:
    \begin{itemize}
        \item For something to be soluble, you need to have favorable forces between them.
        \item Ionic compounds are soluble in water, less soluble in polar solvents, and insoluble in nonpolar solvents.
        \item Organic compounds: $<3$ carbons is soluble, 4-5 carbons is borderline, $\geq 6$ is insoluble. More soluble in organic solvents.
    \end{itemize}
    \item Organic solvents:
    \begin{itemize}
        \item \ce{CH2Cl2} --- methylene chloride.
        \item \ce{HCCl3} --- chloroform.
        \item \ce{H3CCOCH3} --- acetone.
        \item Diethyl ether.
        \item THF.
        \item Cyclohexane.
    \end{itemize}
    \item In TLC, the silica gel is very polar, so polar compounds will not move far up the plate. Nonpolar solvents will drag nonpolar compounds up pretty high.
    \item HOMO and LUMO get closer as conjugation increases.
    \item IR spectroscopy:
    \begin{itemize}
        \item The frequencies absorbed vary based on the type. Higher stretching frequencies for lighter atoms and stronger bonds.
        \item IR radiation causes transitions in vibrational modes of bonds.
        \item The stronger the bond and the lighter the atoms, the faster the vibration of the molecule and the higher the stretching frequency.
        \item The $\Delta E$'s are inherent characteristics of the bonds and nuclei.
        \item Bonds absorb light of characteristic energy, frequency, and wavelength.
        \item We usually report IR spectra in terms of the wavenumber $\bar{\nu}$.
        \item The frequencies absorbed can indicate bond types and functional groups in the molecule.
        \item Anything above 1500 (of wavenumber less than 1500) is called the \textbf{fingerprint region} --- it may not tell you what a molecule is, but it will tell you if two molecules are the same.
        \item Sharp peaks at high wavenumbers are characteristic of \ce{N-H} interactions.
        \item Make a line at $\SI{3000}{\per\centi\meter}$. Things to the right of that indicate aliphatic \ce{C-H}'s. Things to the left indicate $sp^2$ \ce{C-H} groups. Things more to the left indicate $sp$ \ce{C-H} groups.
        \item Not all bonds are visible --- stretching bands must change the dipole. Thus, for example, the \ce{C=H} stretch in trans-but-2-ene is not IR active, but the \ce{C=H} stretch in cis-but-2-ene is IR active.
        \item If you want to substitute \ce{D} for \ce{H}, the peak formerly associated with the \ce{R-H} bond will move lower.
    \end{itemize}
    \begin{table}[h!]
        \centering
        \small
        \renewcommand{\arraystretch}{1.2}
        \begin{tabular}{|l|l|}
            \hline
            \multicolumn{2}{|c|}{\textbf{COMMON ABSORPTIONS}}\\ \hline
            Aromatic \ce{C-C} & Two peaks usually in the range of $\SIrange{1500}{1600}{\per\centi\meter}$\\ \hline
            \ce{C=C} & $\sim\SI{1650}{\per\centi\meter}$\\ \hline
            \ce{C=O} & $\sim\SI{1710}{\per\centi\meter}$ (shifts to $\sim\SI{1735}{\per\centi\meter}$ for esters)\\ \hline
            \ce{C=C} & $\SIrange{2100}{2300}{\per\centi\meter}$\\ \hline
            \ce{C=N} & $\SIrange{2100}{2300}{\per\centi\meter}$\\ \hline
            \ce{C-H} (aldehyde) & Two peaks at $\SI{2170}{\per\centi\meter}$ and $\SI{2810}{\per\centi\meter}$\\ \hline
            $sp^3$ \ce{C-H} & Just to the right of $\SI{3000}{\per\centi\meter}$\\ \hline
            $sp^2$ \ce{C-H} & Just to the left of $\SI{3000}{\per\centi\meter}$\\ \hline
            $sp$ \ce{C-H} & $\sim\SI{3300}{\per\centi\meter}$\\ \hline
            \ce{N-H} & $\sim\SI{3300}{\per\centi\meter}$ (one peak for \ce{-NH-}, two peaks for \ce{-NH2})\\ \hline
            \ce{O-H} (alcohol) & $\sim\SI{3400}{\per\centi\meter}$ (a broad, smooth peak)\\ \hline
            \ce{O-H} (acid) & $\sim\SIrange{2500}{3500}{\per\centi\meter}$ (a very broad, ugly [not smooth] peak)\\ \hline
        \end{tabular}
        \caption{Common IR spectroscopy absorptions.}
        \label{fig:IRabsorptions}
    \end{table}
\end{itemize}




\end{document}