\documentclass[../notes.tex]{subfiles}

\pagestyle{main}
\renewcommand{\chaptermark}[1]{\markboth{\chaptername\ \thechapter\ (#1)}{}}
\setcounter{chapter}{1}

\begin{document}




\chapter{Families of Carbon Compounds / Acids and Bases}
\section{Families of Carbon Compounds}
\begin{itemize}
    \item \marginnote{10/7:}Hydrocarbons:
    \begin{itemize}
        \item Alkanes (\ce{C_$n$H_{$2n+2$}}) and cycloalkanes \ce{C_$n$H_{2$n$}}.
        \item Alkenes (\ce{C_$n$H_{$2n$}}).
        \item Alkynes (\ce{C_$n$H_{$2n-2$}}).
    \end{itemize}
    \item Aromatic:
    \begin{itemize}
        \item Contains a benzene ring.
        \item All bonds $\sim\SI{140}{\angstrom}$.
        \item All carbons $sp^2$.
        \item Planar.
        \item $\pi$ electrons above and below the ring.
        \item Special stabilization.
    \end{itemize}
    \item Covers drawing dipoles.
    \item Polar and nonpolar molecules:
    \begin{itemize}
        \item $\text{Dipole}=\text{distance}\times\text{change between charges}$.
        \item $\mu=r\times Q$
        \item $\SI{1}{\debye}=\SI{3.336e-30}{\coulomb\meter}$.
        \item Analyzes molecules by drawing a Lewis structure, drawing a dipole along each bond, and drawing and labeling a net dipole, if applicable.
        \item Goes through a number of examples.
        \item Acetonitrile is a strong polar solvent.
    \end{itemize}
    \item \textbf{Functional group}: A common arrangement that determines shape, bonding physical and reactivity of organic compounds.
    \item Families of carbon compounds:
    \begin{itemize}
        \item Hydrocarbons: Aliphatic, aromatic.
        \item Methyl, ethyl, propyl, $R=\text{alkyl}$ groups.
        \item Phenyl: \ce{Ph{-}} or \ce{$\phi${-}}.
        \item Benzyl: \ce{Ph-CH2{-}}, \ce{C6H5CH2{-}}, \ce{Bn{-}}
        \item Compounds with \ce{R-Z} where Z is a heteroatom.
        \begin{itemize}
            \item If Z is a halogen X, then the halogroup makes it an alkyl halide or haloalkane.
        \end{itemize}
        \item Alkenyl halide: \ce{X-=}.
        \item Aryl halide: \ce{Ph-X}.
        \item Alcohols or phenols: \ce{R-OH}.
        \item Ether: \ce{R-O-R$'$}.
        \item Amines: \ce{NH2R}, \ce{NHRR$'$}, \ce{NRR$'$R$''$}.
        \item Thiols or mercathols: \ce{R-SH}.
        \item Carbonyl group: \ce{R-CO-R$'$}.
        \item Aldehyde: \ce{R-COH}.
        \item Ketone: \ce{R-CO-R$'$}.
        \item Carboxylic acid derivatives:
        \begin{itemize}
            \item Acid: \ce{R-COOH}.
            \item Ester: \ce{R-COOR$'$}.
            \item Acid chloride: \ce{R-COCl}.
            \item Acid halide: \ce{R-COX}.
            \item Amide: \ce{R-CONH2}.
            \item Acid anhydride: \ce{R-COOCO-R$'$}.
        \end{itemize}
        \item Nitrile: \ce{R-C#N}.
        \item Acrylonitrile: \ce{=-C#N}.
    \end{itemize}
\end{itemize}



\section{Discussion Section}
\begin{itemize}
    \item ACS in-text citations should be in superscripts as a list of number with no brackets or parentheses.
    \item Molecular formulas are \ce{C2H6O}, not \ce{C2H5OH} or \ce{CH3CH2OH}.
    \item Make a table if you have a lot of data to put in (make it readable!).
    \item Distillation:
    \begin{itemize}
        \item We need a boiling chip and stir bar inside the flask.
        \item Vapor comes up from a round-bottomed flask, encounters a rubber stopper and gets diverted through a condenser instead.
        \item Make use of countercurrent exchange and increase pressure by inflowing water in the gravitationally lower portion of the condenser.
        \item Boiling chip is a coarse material with a lof of micropores inside.
        \item The surface energy is reduced when the fluid is inside the micropores; within, it can more easily become a gas.
    \end{itemize}
    \item As the mole fraction $\chi$ of a substance \ce{A} increases\dots
    \item Raoult's law:
    \begin{equation*}
        P_\text{total} = \frac{P_A\chi_A}{P_B\chi_B} = \frac{P_A\chi_A}{P_B(1-\chi_A)}
    \end{equation*}
    \item Dalton's law: The total pressure is equal to the sum of the partial pressures.
\end{itemize}




\end{document}