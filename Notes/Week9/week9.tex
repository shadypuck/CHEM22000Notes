\documentclass[../notes.tex]{subfiles}

\pagestyle{main}
\renewcommand{\chaptermark}[1]{\markboth{\chaptername\ \thechapter\ (#1)}{}}
\setcounter{chapter}{8}

\begin{document}




\chapter{Nucleophilic Substitutions and Elimination}
\section{Nucleophilic Substitutions (cont.)}
\begin{itemize}
    \item \marginnote{11/30:}Choosing between the mechanisms (cont.).
    \begin{enumerate}
        \setcounter{enumi}{3}
        \item Solvent.
        \begin{itemize}
            \item Critical for borderline cases.
            \item The solvent is important for dissolving things/providing an environment for the reaction.
            \item There are two types of solvents: \textbf{protic} and \textbf{aprotic}.
            \item We need to know all of the common solvents (a table will be uploaded).
            \item Key difference between protic and aprotic solvents.
            \begin{itemize}
                \item Protic solvents can do hydrogen bonding with anions (LGs), stabilizing them.
                \item Aprotic solvents cannot do this.
            \end{itemize}
            \item Protic solvents can stabilize \ce{X-}, easing the self-ionization step in S\textsubscript{N}1.
            \item Protic solvents stabilize both the nucleophile and LG in S\textsubscript{N}2.
            \begin{itemize}
                \item With the nucleophile retarded, the rate of S\textsubscript{N}2 goes down.
            \end{itemize}
            \item In an aprotic solvent, the nucleophile is even more reactive.
            \item Take-home message: For secondary alkyl halides (the borderline cases), protic solvents promote S\textsubscript{N}1 and aprotic solvents promote S\textsubscript{N}2.
        \end{itemize}
    \end{enumerate}
    \item \textbf{Protic} (solvent): A solvent with an acidic proton.
    \item \textbf{Aprotic} (solvent): A solvent without an acidic proton.
    \item If you see a nucleophilic substitution-type reaction with just one compound surrounding the arrow, assume it is both the nucleophile \emph{and} the solvent.
    \item \textbf{Allylic} (carbocation): A carbocation on a carbon adjacent to an alkene.
    \begin{itemize}
        \item Extra stable due to resonance stabilization.
    \end{itemize}
\end{itemize}



\section[\texorpdfstring{$\beta$}{TEXT}-Elimination]{\texorpdfstring{$\bm{\beta}$}{TEXT}-Elimination}
\begin{itemize}
    \item $\beta$-elimination is a form of dehydrohalogenation.
    \item General form.
    \begin{center}
        \footnotesize
        \schemestart
            \chemfig{\charge{45:1pt=\tiny${\color{rex}\beta}$}{C}(-[2]H)(-[4])(-[6])-\charge{45:1pt=\tiny${\color{rex}\alpha}$}{C}(-[2])(-)(-[6]X)}
            \arrow{->[base (\ce{B-})]}[,1.3]
            \chemfig{C(-[3])(-[5])=C(-[1])(-[7])}
            \+
            \chemfig{HB}
            \+
            \chemfig{\ce{X-}}
        \schemestop
    \end{center}
    \item Mechanisms.
    \begin{figure}[h!]
        \centering
        \footnotesize
        \begin{subfigure}[b]{\linewidth}
            \centering
            \schemestart
                \chemfig{(-[:120])(-[:-120])(-[:-160])-[@{sb1}]@{Br1}Br}
                \arrow{->[][-\ce{Br-}]}
                \chemfig{\charge{-45:3pt=$\oplus$}{}(-[:120])(-[:-120])-[@{sb2a}]-[@{sb2b}:60]@{H2}H}
                \arrow{->[\chemfig[atom sep=1.4em]{H-[1]@{O3}\charge{[extra sep=1.5pt]45=\:,135=\:}{O}-[7]Et}][-\ce{EtOH2}]}[,1.5]
                \chemfig{(-[:120])(-[:-120])=}
            \schemestop
            \chemmove{
                \draw [rex,semithick,shorten <=2pt,shorten >=2pt] (sb1) to[bend left=90,looseness=3] (Br1);
                \draw [rex,semithick,shorten <=6pt,shorten >=2pt] (O3) to[bend right=45,looseness=1.2] (H2);
                \draw [rex,semithick,shorten <=2pt,shorten >=2pt] (sb2b) to[bend right=60,looseness=1.7] (sb2a);
            }
            \caption{E1.}
            \label{fig:eliminationa}
        \end{subfigure}\\[1em]
        \begin{subfigure}[b]{\linewidth}
            \centering
            \schemestart
                \chemfig{@{B1}\charge{0=\:,45:3pt=$\ominus$}{B}}
                \arrow{0}[,0.5]
                \chemfig{(-[@{sb2a}:120]@{H2}H)(<[:-120]\ce{R^3})(<:[:-160]\ce{R^4})-[@{sb2b}](<:[:60]\ce{R^1})(<[:20]\ce{R^2})(-[@{sb2c}:-60]@{X2}X)}
                \arrow
                \chemfig{(-[:120]\ce{R^4})(-[:-120]\ce{R^3})=(-[:60]\ce{R^1})(-[:-60]\ce{R^2})}
                \+
                \chemfig{HB}
                \+
                \chemfig{\charge{45:3pt=$\ominus$}{X}}
            \schemestop
            \chemmove{
                \draw [rex,semithick,shorten <=6pt,shorten >=2pt] (B1) to[out=0,in=135,looseness=1.5] (H2);
                \draw [rex,semithick,shorten <=2pt,shorten >=2pt] (sb2a) to[bend left=60,looseness=1.7] (sb2b);
                \draw [rex,semithick,shorten <=2pt,shorten >=2pt] (sb2c) to[bend right=60,looseness=1.7] (X2);
            }
            \caption{E2.}
            \label{fig:eliminationb}
        \end{subfigure}
        \caption{Elimination mechanisms.}
        \label{fig:elimination}
    \end{figure}
    \item \textbf{E1}: Unimolecular elimination.
    \begin{itemize}
        \item Not a clean reaction --- E1 and S\textsubscript{N}1 often happen together.
        \item They will not ask us to tell which pathway is more favored.
        \item Features.
        \begin{enumerate}
            \item Tertiary alkyl halides are favored (secondary sometimes).
            \item Protic solvents are needed.
            \item We need a weak base/poor nucleophile.
            \item Selectivity: Determined by the alkene stability; we favor forming the more stable alkene (as per \textbf{Zaitsev's Rule}).
        \end{enumerate}
        \item E1 is not a useful reaction to prepare alkenes from alkyl halides since we get a mixture of products and there are selectivity issues.
    \end{itemize}
    \item \textbf{Zaitsev's Rule}: More substituted alkenes are more stable.
    \begin{itemize}
        \item For secondary carbons, $\text{cis}<\text{trans}<\text{geminal}$ in terms of stability.
        \item Since $sp^2$ is more electronegative than $sp^3$ and \ce{R} is an EDG, more \ce{R} groups can provide more electrons to stabilize the $sp^2$ carbons.
    \end{itemize}
    \item \textbf{E2}: Bimolecular elimination.
    \begin{itemize}
        \item Often very selective, and you can make it very selective.
        \begin{itemize}
            \item The lack of a carbocation intermediate and the fact that it's a concerted mechanism both contribute to the higher yield.
        \end{itemize}
        \item In order to realize E2, the conformation \emph{must} adopt \textbf{anti-periplanar geometry}.
        \begin{itemize}
            \item E2 is a stereospecific reaction.
        \end{itemize}
        \item A bulky base (such as \ce{Bu^{$t$}O-}) is preferred since such a base reduces competition from S\textsubscript{N}2.
    \end{itemize}
    \item \textbf{Anti-periplanar geometry}: Two groups of importance are opposite each other and lie in the same plane.
    \begin{itemize}
        \item Consider the \ce{H} and \ce{X} in Figure \ref{fig:eliminationb}.
    \end{itemize}
    \item Example: Consider \emph{cis}-1-chloro-2-isopropylcyclohexane in solution with \ce{MeONa} and \ce{MeOH}.
    \begin{itemize}
        \item Only the more stable cyclohexane conformation (with \ce{Cl} axial and \ce{Pr^{$i$}} equatorial) has hydrogens anti to the chlorine.
        \item Both of these hydrogens will undergo E2 elimination with the chlorine, and the trisubstituted product will be the major product (as per Zaitsev's rule).
        \item However, if we use $t$-butoxide instead of methoxide, the disubstituted product would be the major product due to sterics.
    \end{itemize}
    \item Example: Consider \emph{trans}-1-chloro-2-isopropylcyclohexane in solution with \ce{MeONa} and \ce{MeOH}.
    \begin{itemize}
        \item Since the less stable conformation is the reactive one, the reaction will still go, but it will be very slow.
    \end{itemize}
    \item Take-home summary: For E2, the first priority is anti-periplanar, and then Zaitsev.
    \item \marginnote{12/6:}Deciding between S\textsubscript{N}2 and E2 in secondary cases.
    \begin{itemize}
        \item When you have a strong base, E2 is favored.
        \begin{itemize}
            \item Examples: \ce{OH-}, \ce{MeO-}, \ce{EtO-}, \ce{Bu^{$t$}O-}.
        \end{itemize}
        \item When you have a good nucleophile that is not too basic, S\textsubscript{N}2 is favored.
        \begin{itemize}
            \item Examples: \ce{Br-}, \ce{I-}, \ce{RS-}, \ce{N#C-}, \ce{N3-}, \ce{PPh3}.
        \end{itemize}
    \end{itemize}
    \item Deciding between E2 and E1/S\textsubscript{N}1.
    \begin{itemize}
        \item For E2, the role of the base is critical --- without a strong base, it will not take place.
        \item For E1, the role of the base is not important; ionization is more important.
        \item Since ionization is a very slow process, if there is competition between E2 and E1 and a strong base is present, E2 will usually win out because it's so much faster.
        \item Primary alkyl halides lead to E2 only.
        \item Secondary and tertiary alkyl halides lead to E2 in the presence of a strong base, and E1/S\textsubscript{N}1 in the presence of a weak base/solvent.
    \end{itemize}
\end{itemize}



\section{Alkyl Halide Equivalents}
\begin{itemize}
    \item Other species that can behave with the above chemistry.
    \item Consider the following sulfonate ester.
    \begin{figure}[h!]
        \centering
        \footnotesize
        \chemfig{R-O-S(=[2]O)(=[6]O)-*6(=-=(-CH_3)-=-)}
        \caption{A sulfonate ester.}
        \label{fig:sulfonateEster}
    \end{figure}
    \begin{itemize}
        \item Often abbreviated \ce{OTs} and called tosylate.
    \end{itemize}
    \item This species is important because we can make it from alcohols with stereoretension.
    \begin{figure}[H]
        \centering
        \footnotesize
        \schemestart
            \chemfig{\ce{R^1}-[:30](<[2]OH)-[:-30]\ce{R^2}}
            \arrow{->[\ce{TsCl}][\chemfig[atom sep=1.4em]{*6(=N-=-=-)}]}[,1.3]
            \chemfig{\ce{R^1}-[:30](<[2]OTs)-[:-30]\ce{R^2}}
        \schemestop
        \caption{Making tosylate species.}
        \label{fig:tosylateSpecies}
    \end{figure}
    \begin{itemize}
        \item The species below the arrow above is called pyridine and is often abbreviated Py.
        \item The mechanism of the above reaction is not needed.
    \end{itemize}
    \item You can then hit the product in Figure \ref{fig:tosylateSpecies} with a nucleophile to perform S\textsubscript{N}2.
    \item Note that like \ce{RX}, \ce{ROTs} can also undergo $\beta$-elimination (such as E2).
\end{itemize}



\section{Alkyne Synthesis}
\begin{itemize}
    \item You can form alkynes from simpler alkynes, alkenes, and ketones (though we don't have to know the last one for this class).
    \item Alkylation of acetylides with \ce{RX} via S\textsubscript{N}2.
    \begin{equation*}
        \ce{H-C#C-H ->[1. NaNH2][2. R-Br] H-C#C-R ->[1. NaNH2][2. R$'$-Br] R$'$-C#C-R}
    \end{equation*}
    \begin{itemize}
        \item Start with an acetylide such as \ce{NaC#C-H}.
        \item React it with an alkyl bromide (\ce{RBr}) in THF to attach it to that alkyl species at the former bromium site with inverted stereochemistry (yielding \ce{RC#C-H} and \ce{NaBr}).
        \item React the terminal alkyne species with a strong base (e.g., \ce{NaH}, \ce{NaNH2}, \ce{LDA}) to generate a species such as \ce{NaC#CR}.
        \item React this with another alkyl bromide (\ce{R$'$Br}) to yield the final \ce{RC#CR$'$} species.
    \end{itemize}
    \item \textbf{Synthesis}: Making a large and more useful molecule from readily available small molecules.
    \begin{itemize}
        \item Using alkynes is a very important approach to make carbon-carbon bonds.
    \end{itemize}
    \item From alkenes.
    \begin{figure}[h!]
        \centering
        \footnotesize
        \begin{subfigure}[b]{\linewidth}
            \centering
            \schemestart
                \chemfig{R-CH=CH-R'}
                \arrow{->[\ce{Br2}]}
                \chemfig{R-C(-[2]H)(-[6]Br)-C(-[2]Br)(-[6]H)-R'}
                \arrow{->[\ce{2NaNH2}]}[,1.4]
                \chemfig{R-C~C-R'}
            \schemestop
            \caption{Internal alkene.}
            \label{fig:alkyneAlkeneSynthesisa}
        \end{subfigure}\\[1em]
        \begin{subfigure}[b]{\linewidth}
            \centering
            \schemestart
                \chemfig{R-CH=CH_2}
                \arrow{->[\ce{Br2}]}
                \chemfig{R-C(-[2]H)(-[6]Br)-C(-[2]Br)(-[6]H)-H}
                \arrow{->[\ce{3NaNH2}]}[,1.4]
                \chemfig{R-C~CNa}
                \arrow{->[\ce{H2O}]}
                \chemfig{R-C~C-H}
            \schemestop
            \caption{Terminal alkene.}
            \label{fig:alkyneAlkeneSynthesisb}
        \end{subfigure}
        \caption{Synthesis of alkynes from alkenes.}
        \label{fig:alkyneAlkeneSynthesis}
    \end{figure}
    \begin{itemize}
        \item The elimination mechanism for the two pairs of \ce{HBr} in Figure \ref{fig:alkyneAlkeneSynthesisa} is different from E1 and E2, and we do not need to know it.
        \item In Figure \ref{fig:alkyneAlkeneSynthesisb}, we need the third equivalent of base in the second step because once an alkyne species is formed, its acidic proton will react with any base in solution. Thus, if we used only two equivalents, some of the reactant would get converted all of the way to the \ce{R-C#CNa} species, and some would not get converted at all. Therefore, we push all of the reactant to be converted, and then work with the product, quenching with \ce{H2O} to get our final desired product.
        \item Note that various byproducts are generated that are not shown (they are the predictable ones, though).
        \item We can also use chloride here.
    \end{itemize}
\end{itemize}



\section{Multi-Step Synthesis}
\begin{itemize}
    \item These problems are the core of organic chemistry, using both our imagination and our knowledge to construct \emph{a} right answer (there are often more than one).
    \item Tip: Think backwards!
    \begin{itemize}
        \item Formally, "retro-synthetic analysis," as coined by E. J. Corey, a Nobel laureate at Harvard.
    \end{itemize}
    \item Example: Construct cyclohexane-1,2-diol from cyclohexanol.
    \begin{equation*}
        \ce{C6H11OH ->[TsCl, Py] C6H11OTs ->[Bu^{$t$}O-][-Bu^{$t$}OH, OTs-] C6H10 ->[1. OsO4][2. NaHSO3] C6H10(OH)2}
    \end{equation*}
    \item More examples given.
\end{itemize}




\end{document}