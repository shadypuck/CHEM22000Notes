\documentclass[../notes.tex]{subfiles}

\pagestyle{main}
\renewcommand{\chaptermark}[1]{\markboth{\chaptername\ \thechapter\ (#1)}{}}
\setcounter{chapter}{4}

\begin{document}




\chapter{Stereochemistry / Reactions of Alkenes}
\section{Stereochemistry Nomenclature and Intro to Alkenes}
\begin{itemize}
    \item \marginnote{10/28:}Last time:
    \begin{itemize}
        \item Chapter 4 (di-substituted cyclohexanes, bicyclic/polycyclic rings).
        \item Chapter 5 (stereochemistry, stereoisomers/chirality center, chirality tests [plane/center symmetry], R/S system [nomenclature; a very important survival skill for this class], physical properties of enantiomers, achiral environment: same; chiral environment: different, rotate plane polarized light [left (-) levorotatory, right (+) dextrorotatory, racemic ($\pm$) mixture], $[\alpha]_D^{2t}$ specific rotation, enantiometric excess, Fischer projections [less important]).
    \end{itemize}
    \item Intro to Guangbin Dong.
    \item The textbook is only a guide --- follow the lecture. Also, if you see discrepancies either with the book or with Piccirilli, bring them up.
    \item Capital $E$ is energy and lowercase $e$ is electrons in this class.
    \item Reading assignment: Chapter 5.12, 5.14, 4.5 (review), 4.17, 7.1-7.4.
    \item Nongraded homework: 6.39, 5.40, 5.46, 5.48, 7.1, 7.17, 7.18, 30a-d.
    \item Multiple stereocenters.
    \item Worked example: Naming (2S,3R)-2-chloro-3-iodobutane.
    \begin{itemize}
        \item $n=2$ stereocenters yields at most $2^n=2^2=4$ stereoisomers.
        \item Draws all enantiomers.
    \end{itemize}
    \item \textbf{Diastereomers}: Have at least 2 stereocenters, same formula, same connectivity, but different orientation in space and are not mirror images of each other.
    \begin{itemize}
        \item Special case: cis/trans isomerism.
    \end{itemize}
    \item Compounds with 2 stereocenters don't always have 4 stereoisomers.
    \item Example: Tartaric acid, for which the (2S,3R) compound is superimposable on the (2R,3S) compound. This stereoisomer is a \textbf{meso} isomer and not a chiral molecule (there exists a plane of symmetry).
    \begin{itemize}
        \item Thus, a molecule can be achiral even if it has a chiral center!
    \end{itemize}
    \item \textbf{Meso} (compound): A compound with chiral stereocenters and an internal plane of symmetry.
    \begin{itemize}
        \item The following compound (two conformers shown) is also meso because the Conformer 2 has the desired plane of symmetry.
        \begin{figure}[h!]
            \centering
            \footnotesize
            \begin{subfigure}[b]{0.25\linewidth}
                \centering
                \chemfig{-[:-60](<[:-120]Cl)-(>:[:60]Cl)(<[:20]H)-[:-60]}
                \caption{Conformer 1.}
                \label{fig:mesoCompounda}
            \end{subfigure}
            \begin{subfigure}[b]{0.25\linewidth}
                \centering
                \chemfig{-[:-60](<[:-120]Cl)-(<[:-60]Cl)-[:60]}
                \caption{Conformer 2.}
                \label{fig:mesoCompoundb}
            \end{subfigure}
            \caption{Meso compounds.}
            \label{fig:mesoCompound}
        \end{figure}
    \end{itemize}
    \item Alkenes:
    \begin{itemize}
        \item One of the most important functional groups.
        \item Basic industry materials: polyethylene and polypropylene.
        \item Biological systems: Fatty acids, vitamins (Vitamin A), drugs/natural products, important building block in other FGs.
    \end{itemize}
    \item Alkenes have a $\pi$ bond --- $2s$, $2p_x$, and $2p_y$ get hybridized into $sp^2$ orbitals, and $2p_z$ forms the $\pi$ bond.
    \begin{itemize}
        \item The $\pi$ bond leads to the enforced coplanar geometry of the alkanes.
        \item The bond energy of a \ce{C=C} bond is significantly greater than that of a \ce{C-C} bond.
    \end{itemize}
    \item Alkene cis/trans isomerism.
    \begin{itemize}
        \item Trans is more stable b/c of steric hindrance.
        \item E/Z system.
    \end{itemize}
\end{itemize}




\end{document}