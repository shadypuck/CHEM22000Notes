\documentclass[../notes.tex]{subfiles}

\pagestyle{main}
\renewcommand{\chaptermark}[1]{\markboth{\chaptername\ \thechapter\ (#1)}{}}
\setcounter{chapter}{2}

\begin{document}




\chapter{Nomenclature and Conformations of Alkanes and Cycloalkanes}
\section{Conformers}
\begin{itemize}
    \item \marginnote{10/21:}\textbf{Conformational isomers}: Groups connected by single bonds undergo rotation resulting in different molecular conformations. \emph{Also known as} \textbf{conformers}.
    \begin{itemize}
        \item These are transient states.
    \end{itemize}
    \item \textbf{Conformational analysis}: The process of understanding how the conformation relates to the energy of the molecule.
    \item Newman projections and the sawhorse model.
    \item Staggered to eclipsed ethane conformations: $\Delta E=\SI[per-mode=symbol]{12}{\kilo\joule\per\mole}$.
    \begin{itemize}
        \item $\text{Rate}=\SI{5e10}{\hertz}$.
    \end{itemize}
    \item \textbf{Torsional strain}: Repulsive interactions (steric hindrance) between the clouds of electrons of bonded groups.
    \item Goes through butane conformations.
    \begin{itemize}
        \item \textbf{Gauche} vs. \textbf{anti} methyl groups.
    \end{itemize}
    \item \textbf{Ring strain}: The combination of angle strain and torsional strain in a cycloalkane.
    \item Puckering of cyclobutane relieves some of the torsional strain.
    \item Puckering of cyclopentane relieves some torsional strain \emph{and} angle strain.
    \item Cyclohexane has chair and boat conformations.
    \begin{itemize}
        \item Goes through Newman projections for each.
    \end{itemize}
    \begin{figure}[h!]
        \centering
        \footnotesize
        \chemfig{?(-[:60]H)<[:-40]-[,,,,line width=3pt]>[:40](-[:120]H)>:[:-160,0.9]-[4,0.9]?[,6]}
        \caption{Flagpole interactions.}
        \label{fig:flagpoleInteraction}
    \end{figure}
    \item \textbf{Flagpole interaction}: The interaction between the two hydrogens on opposite carbons in cyclohexane that bend toward each other.
    \item Axial and equatorial positions.
    \item In a ring flip, axial and equatorial hydrogens invert.
    \item When you have substituents on a ring, you have 1,3-diaxial interactions.
    \item \marginnote{10/26:}Covers disubstituted cyclohexanes and bicyclic/polycyclic alkanes.
\end{itemize}




\end{document}