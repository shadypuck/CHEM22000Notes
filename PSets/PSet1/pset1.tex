\documentclass[../psets.tex]{subfiles}

\pagestyle{main}
\renewcommand{\leftmark}{Problem Set \thesection}

\begin{document}




\section{Bonding and Molecular Structure}
\begin{enumerate}
    \setlist[enumerate,3]{label={(\roman*)}}
    \item \marginnote{10/14:}Write a Lewis structure for each of the following compounds and indicate whether the bonding is nonpolar covalent, polar covalent, or ionic. Assume that a difference in electronegativity greater than 1.7 corresponds to a bond that is considered predominantly ionic.
    \begin{enumerate}
        \item \ce{HCl}.
        \begin{proof}[Answer]
            ${\color{white}hi}$
            \begin{center}
                \chemfig{H-\charge{90=\:,0=\:,-90=\:}{Cl}}
            \end{center}
            Polar covalent.
        \end{proof}
        \item \ce{C2H6}.
        \begin{proof}[Answer]
            ${\color{white}hi}$
            \begin{center}
                \chemfig{H-C(-[2]H)(-[6]H)-C(-[2]H)(-[6]H)-H}
            \end{center}
            Nonpolar covalent.
        \end{proof}
        \item \ce{NaBr}.
        \begin{proof}[Answer]
            ${\color{white}hi}$
            \begin{center}
                \chemfig{\charge{45:3pt=$+$}{[Na]}-[,0.5,,,white]\charge{45[xshift=6pt,yshift=2pt]=$-$}{[\ \ \charge{0=\:,90=\:,180=\:,270=\:}{Br}\ \ ]}}
            \end{center}
            Ionic.
        \end{proof}
        \item \ce{CH3I}.
        \begin{proof}[Answer]
            ${\color{white}hi}$
            \begin{center}
                \chemfig{H-C(-[2]H)(-[6]H)-\charge{90=\:,0=\:,-90=\:}{I}}
            \end{center}
            Nonpolar covalent.
        \end{proof}
        \item \ce{H2S}.
        \begin{proof}[Answer]
            ${\color{white}hi}$
            \begin{center}
                \chemfig{H-\charge{90=\:,-90=\:}{S}-H}
            \end{center}
            Nonpolar covalent.
        \end{proof}
        \item \ce{N2H4}.
        \begin{proof}[Answer]
            ${\color{white}hi}$
            \begin{center}
                \chemfig{H-\charge{90=\:}{N}(-[6]H)-\charge{90=\:}{N}(-[6]H)-H}
            \end{center}
            Polar covalent.
        \end{proof}
        \item \ce{CsF}.
        \begin{proof}[Answer]
            ${\color{white}hi}$
            \begin{center}
                \chemfig{\charge{45:3pt=$+$}{[Cs]}-[,0.5,,,white]\charge{45[xshift=6pt,yshift=2pt]=$-$}{[\ \ \charge{0=\:,90=\:,180=\:,270=\:}{F}\ \ ]}}
            \end{center}
            Ionic.
        \end{proof}
    \end{enumerate}
    \item For the following covalent bonds\dots
    \begin{enumerate}
        \item Use the symbols $\delta^+$ and $\delta^-$ to indicate the direction of polarity (if any).
        \begin{enumerate}
            \item \ce{C-F}.
            \begin{proof}[Answer]
                ${\color{white}hi}$
                \begin{center}
                    \chemfig{\charge{135[xshift=-1pt]=$\delta^+$}{C}-\charge{45[xshift=5pt]=$\delta^-$}{F}}
                \end{center}
            \end{proof}
            \item \ce{N-Br}.
            \begin{proof}[Answer]
                Nonpolar.
            \end{proof}
            \item \ce{B-C}.
            \begin{proof}[Answer]
                ${\color{white}hi}$
                \begin{center}
                    \chemfig{\charge{135[xshift=-1pt]=$\delta^+$}{B}-\charge{45[xshift=5pt]=$\delta^-$}{C}}
                \end{center}
            \end{proof}
            \item \ce{Si-H}.
            \begin{proof}[Answer]
                Nonpolar.
            \end{proof}
        \end{enumerate}
        \item Rank the following covalent bonds in order of \emph{increasing} polarity.
        \begin{enumerate}
            \item \ce{C-H}, \ce{O-H}, \ce{N-H}.
            \begin{proof}[Answer]
                \begin{equation*}
                    \ce{C-H} < \ce{N-H} < \ce{O-H}
                \end{equation*}
            \end{proof}
            \item \ce{C-N}, \ce{C-O}, \ce{B-O}.
            \begin{proof}[Answer]
                \begin{equation*}
                    \ce{C-N} < \ce{C-O} < \ce{B-O}
                \end{equation*}
            \end{proof}
            \item \ce{C-P}, \ce{C-S}, \ce{C-N}.
            \begin{proof}[Answer]
                \begin{equation*}
                    \ce{C-S} < \ce{C-P} < \ce{C-N}
                \end{equation*}
            \end{proof}
        \end{enumerate}
    \end{enumerate}
    \item Formal charge.
    \begin{enumerate}
        \item Consider the oxygen atom in the structures below and determine if it has a formal charge. If so, label it on the molecule.
        \begin{proof}[Answer]
            ${\color{white}hi}$
            \begin{figure}[H]
                \centering
                \small
                \begin{subfigure}[b]{0.19\linewidth}
                    \centering
                    \chemfig{\charge{90=\:,45:3pt=$\oplus$}{O}(-H)(-[4]H)(-[6]H)}
                \end{subfigure}
                \begin{subfigure}[b]{0.19\linewidth}
                    \centering
                    \chemfig{H-\charge{0=\:,90=\:,270=\:,45:3pt=$\ominus$}{O}}
                \end{subfigure}
                \begin{subfigure}[b]{0.19\linewidth}
                    \centering
                    \chemfig{\charge{90=\:,45:3pt=$\oplus$}{O}(-H)=[4](-[:120]H)(-[:-120]H)}
                \end{subfigure}
                \begin{subfigure}[b]{0.19\linewidth}
                    \centering
                    \chemfig{\charge{[extra sep=1.5pt]135=\:,-135=\:}{O}=C=\charge{[extra sep=1.5pt]45=\:,-45=\:}{O}}
                \end{subfigure}
                \begin{subfigure}[b]{0.19\linewidth}
                    \centering
                    \chemfig{-[:30](=[2]\charge{[extra sep=1.5pt]45=\:,135=\:}{O})-[:-30]\charge{0=\:,90=\:,270=\:,45:3pt=$\ominus$}{O}}
                \end{subfigure}
            \end{figure}
        \end{proof}
        \item Consider the nitrogen atom in the structures below and determine if it has a formal charge. If so, label it on the molecule.
        \begin{proof}[Answer]
            ${\color{white}hi}$
            \begin{figure}[H]
                \centering
                \small
                \begin{subfigure}[b]{0.19\linewidth}
                    \centering
                    \chemfig{\charge{45:3pt=$\oplus$}{N}(=CH_2)(-[:120]H_3C)(-[:-120]H)}
                \end{subfigure}
                \begin{subfigure}[b]{0.19\linewidth}
                    \centering
                    \chemfig{H_3C-\charge{90=\:,270=\:,45:3pt=$\ominus$}{N}-H}
                \end{subfigure}
                \begin{subfigure}[b]{0.19\linewidth}
                    \centering
                    \chemfig{\charge{270=\:}{N}(-\charge{90=\:,270=\:}{O}H)(-[2]H)(-[4]H)}
                \end{subfigure}
                \begin{subfigure}[b]{0.19\linewidth}
                    \centering
                    \chemfig{H_3C-\charge{90=\:}{N}H_2}
                \end{subfigure}
                \begin{subfigure}[b]{0.19\linewidth}
                    \centering
                    \chemfig{\charge{45:3pt=$\oplus$}{N}(-CH_3)(-[2]CH_3)(-[4]H_3C)(-[6]CH_3)}
                \end{subfigure}
            \end{figure}
        \end{proof}
    \end{enumerate}
    \item Draw Lewis structures and resonance structures (if any) that satisfy the octet rule for each of the following ions with all valence electrons and formal charges clearly noted.
    \begin{enumerate}
        \item \ce{NH2-}.
        \begin{proof}[Answer]
            ${\color{white}hi}$
            \begin{center}
                \schemestart
                    \chemleft{[}
                        \chemfig{H-\charge{90=\:,-90=\:,45:3pt=$\ominus$}{N}-H}
                    \chemright{]^-}
                \schemestop
            \end{center}
        \end{proof}
        \item \ce{NO2-}.
        \begin{proof}[Answer]
            ${\color{white}hi}$
            \begin{center}
                \schemestart
                    \chemleft{[}
                        \chemfig{\charge{[extra sep=1.5pt]135=\:,-135=\:}{O}=\charge{90=\:}{N}-\charge{90=\:,0=\:,-90=\:,45:3pt=$\ominus$}{O}}
                    \chemright{]^-}
                    \arrow{<->}
                    \chemleft{[}
                        \chemfig{\charge{90=\:,180=\:,270=\:,45:3pt=$\ominus$}{O}-\charge{90=\:}{N}=\charge{[extra sep=1.5pt]45=\:,-45=\:}{O}}
                    \chemright{]^-}
                \schemestop
            \end{center}
        \end{proof}
        \item \ce{ClO-}.
        \begin{proof}[Answer]
            ${\color{white}hi}$
            \begin{center}
                \schemestart
                    \chemleft{[}
                        \chemfig{\charge{90=\:,180=\:,270=\:}{Cl}-\charge{90=\:,0=\:,-90=\:,45:3pt=$\ominus$}{O}}
                    \chemright{]^-}
                \schemestop
            \end{center}
        \end{proof}
        \item \ce{HCOO-}.
        \begin{proof}[Answer]
            ${\color{white}hi}$
            \begin{center}
                \schemestart
                    \chemleft{[}
                        \chemfig{H-C(-[:60]\charge{0=\:,90=\:,180=\:,45:3pt=$\ominus$}{O})(=[:-60]\charge{0=\:,-90=\:}{O})}
                    \chemright{]^-}
                    \arrow{<->}
                    \chemleft{[}
                        \chemfig{H-C(-[:-60]\charge{0=\:,-90=\:,180=\:,45:3pt=$\ominus$}{O})(=[:60]\charge{0=\:,90=\:}{O})}
                    \chemright{]^-}
                \schemestop
            \end{center}
        \end{proof}
        \item \ce{BH4-}.
        \begin{proof}[Answer]
            ${\color{white}hi}$
            \begin{center}
                \schemestart
                    \chemleft{[}
                        \chemfig{\charge{45:3pt=$\ominus$}{B}(-H)(-[2]H)(-[4]H)(-[6]H)}
                    \chemright{]^-}
                \schemestop
            \end{center}
        \end{proof}
        \item \ce{CH3CH2CO2H}.
        \begin{proof}[Answer]
            ${\color{white}hi}$
            \begin{center}
                \chemfig{H-C(-[2]H)(-[6]H)-C(-[2]H)(-[6]H)-C(=[2]\charge{[extra sep=1.5pt]45=\:,135=\:}{O})-\charge{90=\:,-90=\:}{O}-H}
            \end{center}
        \end{proof}
        \item \ce{O3}.
        \begin{proof}[Answer]
            ${\color{white}hi}$
            \begin{center}
                \schemestart
                    \chemfig{\charge{[extra sep=1.5pt]135=\:,-135=\:}{O}=\charge{90=\:,45:3pt=$\oplus$}{O}-\charge{90=\:,0=\:,-90=\:,45:3pt=$\ominus$}{O}}
                    \arrow{<->}
                    \chemfig{\charge{90=\:,180=\:,270=\:,45:3pt=$\ominus$}{O}-\charge{90=\:,45:3pt=$\oplus$}{O}=\charge{[extra sep=1.5pt]45=\:,-45=\:}{O}}
                \schemestop
            \end{center}
        \end{proof}
        \item \ce{CH2N2}.
        \begin{proof}[Answer]
            ${\color{white}hi}$
            \begin{center}
                \chemfig{C(-[:120]H)(-[:-120]H)=\charge{45:3pt=$\oplus$}{N}=\charge{90=\:,-90=\:,45:3pt=$\ominus$}{N}}
            \end{center}
        \end{proof}
    \end{enumerate}
    \item For the following chemical species, draw a resonance structure that satisfies the octet rule. Indicate whether you expect it to be a major or minor contributor to the actual structure of the species and briefly state your reasoning. Use curved arrows to clearly show how the structure converts to another structure (if any).
    \begin{enumerate}[itemsep=1.5em]
        \item   \schemestart
                    \chemfig{[:18]*5(--O-(=O)--)}
                    \arrow{<->}
                \schemestop
        \begin{proof}[Answer]
            ${\color{white}hi}$
            \begin{center}
                \schemestart
                    \chemfig{[:18]*5(--@{O1}\charge{[extra sep=1.5pt]45=\:,-45=\:}{O}-[@{sb}](=[@{db}]@{O2}\charge{[extra sep=1.5pt]45=\:,135=\:}{O})--)}
                    \arrow{<->}
                    \chemfig{[:18]*5(--\charge{0=\:,45:3pt=$\oplus$}{O}=(-\charge{0=\:,90=\:,180=\:,45:3pt=$\ominus$}{O})--)}
                \schemestop
                \chemmove[shorten <=3pt,shorten >=1pt,every path/.append style={looseness=2}]{
                    \draw (db) to[out=0,in=-20] (O2);
                    \draw [shorten <=5pt] (O1) to[out=45,in=50] (sb);
                }
            \end{center}
            The right structure will be a minor contributor because it has formal charges while the original one doesn't.
        \end{proof}
        \item   \schemestart
                    \chemfig{-[:30](=[2]O)-[:-30]\charge{0:3pt=$\ominus$}{}}
                    \arrow{<->}
                \schemestop
        \begin{proof}[Answer]
            ${\color{white}hi}$
            \begin{center}
                \schemestart
                    \chemleft{[}
                        \chemfig{-[:30](=[@{db}2]@{O}\charge{[extra sep=1.5pt]45=\:,135=\:}{O})-[@{sb}:-30]@{C}\charge{0:3pt=$\ominus$}{}}
                    \chemright{]^-}
                    \arrow{<->}
                    \chemleft{[}
                        \chemfig{-[:30](-[2]\charge{0=\:,90=\:,180=\:,45:3pt=$\ominus$}{O})=[:-30]}
                    \chemright{]^-}
                \schemestop
                \chemmove[shorten <=3pt,shorten >=1pt,every path/.append style={looseness=2}]{
                    \draw (C) to[bend right=70] (sb);
                    \draw (db) to[bend right=70] (O);
                }
            \end{center}
            The right structure will be a major contributor because the negative formal charge is on the more electronegative atom.
        \end{proof}
        \item   \schemestart
                    \chemfig{-[:30]-[:-30]-[:30]\charge{90:3pt=$\ominus$}{}-[:-30]~[:-30]N}
                    \arrow{<->}
                \schemestop
        \begin{proof}[Answer]
            ${\color{white}hi}$
            \begin{center}
                \schemestart
                    \chemleft{[}
                        \chemfig{-[:30]-[:-30]-[:30]@{C}\charge{90:3pt=$\ominus$}{}-[@{sb}:-30]~[@{tb}:-30]@{N}\charge{0=\:}{N}}
                    \chemright{]^-}
                    \arrow{<->}
                    \chemleft{[}
                        \chemfig{-[:30]-[:-30]-[:30]=_[:-30]=_[:-30]\charge{0=\:,-90=\:,45:3pt=$\ominus$}{N}}
                    \chemright{]^-}
                \schemestop
                \chemmove[shorten <=3pt,shorten >=1pt,every path/.append style={looseness=2}]{
                    \draw (C) to[bend left=70] (sb);
                    \draw [shorten <=4pt] (tb) to[bend left=70] (N);
                }
            \end{center}
            The right structure will be a major contributor because the negtive formal charge is on the more electronegative atom.
        \end{proof}
        \item   \schemestart
                    \chemfig{\charge{135=$\oplus$}{N}(-[:-150]\charge{135=$\ominus$}{O})(=[2]O)(-[:-30]\charge{-90:3pt=$\ominus$}{}-[:30]-[:-30])}
                    \arrow{<->}
                \schemestop
        \begin{proof}[Answer]
            ${\color{white}hi}$
            \begin{center}
                \schemestart
                    \chemleft{[}
                        \chemfig{\charge{135:3pt=$\oplus$}{N}(-[:-150]\charge{90=\:,180=\:,270=\:,135:3pt=$\ominus$}{O})(=[@{db}2]@{O}\charge{[extra sep=1.5pt]45=\:,135=\:}{O})(-[@{sb}:-30]@{C}\charge{-90:3pt=$\ominus$}{}-[:30]-[:-30])}
                    \chemright{]^-}
                    \arrow{<->}
                    \chemleft{[}
                        \chemfig{\charge{135:3pt=$\oplus$}{N}(-[:-150]\charge{90=\:,180=\:,270=\:,135:3pt=$\ominus$}{O})(-[2]\charge{0=\:,90=\:,180=\:,45:3pt=$\ominus$}{O})(=^[:-30]-[:30]-[:-30])}
                    \chemright{]^-}
                \schemestop
                \chemmove[shorten <=3pt,shorten >=1pt,every path/.append style={looseness=2}]{
                    \draw (C) to[bend left=70,looseness=2.5] (sb);
                    \draw (db) to[bend right=70] (O);
                }
            \end{center}
            The right structure will be a major contributor because the negative formal charges are on the more electronegative atoms.
        \end{proof}
    \end{enumerate}
    \item Draw all four constitutional isomers of \ce{C4H9Br} using bond-line formulas.
    \begin{proof}[Answer]
        ${\color{white}hi}$
        \begin{figure}[h!]
            \centering
            \begin{subfigure}[b]{0.2\linewidth}
                \centering
                \chemfig{Br-[:-30]-[:30]-[:-30]-[:30]}
            \end{subfigure}
            \begin{subfigure}[b]{0.2\linewidth}
                \centering
                \chemfig{-[:30](-[2]Br)-[:-30]-[:30]}
            \end{subfigure}
            \begin{subfigure}[b]{0.2\linewidth}
                \centering
                \chemfig{Br-[:-30]-[:30](-[2])-[:-30]}
            \end{subfigure}
            \begin{subfigure}[b]{0.2\linewidth}
                \centering
                \chemfig{-[:30](-[:70])(-[:110]Br)-[:-30]}
            \end{subfigure}
        \end{figure}
    \end{proof}
    \item For each of the following condensed structures: (i) draw the corresponding Lewis structures, (ii) provide the hybridization to all carbon atoms, and (iii) draw individual $p$ orbitals for all the $\pi$ bonds with directions clearly indicated.
    \begin{enumerate}
        \item \ce{CH2CHCHCH2}.
        \begin{proof}[Answer]
            (i) / (iii):
            \begin{center}
                \chemfig{@{C1}C(-[:120]H)(-[:-120]H)=@{C2}C(-[:60]H)-[:-60]@{C3}C(-[:-120]H)=@{C4}C(-[:60]H)(-[:-60]H)}
                \chemmove{
                    \draw [-,rex,line width=2mm] ([yshift=5mm]C1.90) to[bend left=30] ([yshift=5mm]C2.90);
                    \draw [-,rey,line width=2mm] ([yshift=-5mm]C1.-90) to[bend right=30] ([yshift=-5mm]C2.-90);
                    \draw [-,rex,line width=2mm] ([yshift=5mm]C3.90) to[bend left=30] ([yshift=5mm]C4.90);
                    \draw [-,rey,line width=2mm] ([yshift=-5mm]C3.-90) to[bend right=30] ([yshift=-5mm]C4.-90);
                    %
                    %
                    \filldraw [-,shorten <=2pt,shorten >=2pt,semithick,draw=red,fill=rex] (C1.110) to[bend left=110,looseness=30] (C1.70);
                    \filldraw [-,shorten <=2pt,shorten >=2pt,semithick,draw=red,fill=rey] (C1.-70) to[bend left=110,looseness=30] (C1.-110);
                    %
                    \filldraw [-,shorten <=2pt,shorten >=2pt,semithick,draw=red,fill=rex] (C2.110) to[bend left=110,looseness=30] (C2.70);
                    \filldraw [-,shorten <=2pt,shorten >=2pt,semithick,draw=red,fill=rey] (C2.-70) to[bend left=110,looseness=30] (C2.-110);
                    %
                    \filldraw [-,shorten <=2pt,shorten >=2pt,semithick,draw=red,fill=rex] (C3.110) to[bend left=110,looseness=30] (C3.70);
                    \filldraw [-,shorten <=2pt,shorten >=2pt,semithick,draw=red,fill=rey] (C3.-70) to[bend left=110,looseness=30] (C3.-110);
                    %
                    \filldraw [-,shorten <=2pt,shorten >=2pt,semithick,draw=red,fill=rex] (C4.110) to[bend left=110,looseness=30] (C4.70);
                    \filldraw [-,shorten <=2pt,shorten >=2pt,semithick,draw=red,fill=rey] (C4.-70) to[bend left=110,looseness=30] (C4.-110);
                }
            \end{center}
            (ii): Every carbon atom is $sp^2$.\par
        \end{proof}
        \item \ce{CH3CCCH3}.
        \begin{proof}[Answer]
            (i) / (iii):
            \begin{center}
                \chemfig{H-C(-[2]H)(-[6]H)-@{C2}C~@{C3}C-C(-[2]H)(-[6]H)-H}
                \chemmove{
                    \draw [-,rey,line width=2mm] ([xshift=4mm,yshift=4mm]C2.45) to[bend left=30] ([xshift=4mm,yshift=4mm]C3.45);
                    \draw [-,rey,line width=2mm] ([yshift=-5mm]C2.-90) to[bend right=30] ([yshift=-5mm]C3.-90);
                    %
                    \filldraw [-,shorten <=2pt,shorten >=2pt,semithick,draw=red,fill=rey] (C2.65) to[bend left=110,looseness=30] (C2.25);
                    \filldraw [-,shorten <=2pt,shorten >=2pt,semithick,draw=red,fill=rey] (C3.65) to[bend left=110,looseness=30] (C3.25);
                    %
                    \draw [-,rex,line width=2mm] ([yshift=5mm]C2.90) to[bend left=30] ([yshift=5mm]C3.90);
                    %
                    \filldraw [-,shorten <=2pt,shorten >=2pt,semithick,draw=red,fill=rex] (C2.110) to[bend left=110,looseness=30] (C2.70);
                    \filldraw [-,shorten <=2pt,shorten >=2pt,semithick,draw=red,fill=rey] (C2.-70) to[bend left=110,looseness=30] (C2.-110);
                    %
                    \filldraw [-,shorten <=2pt,shorten >=2pt,semithick,draw=red,fill=rex] (C3.110) to[bend left=110,looseness=30] (C3.70);
                    \filldraw [-,shorten <=2pt,shorten >=2pt,semithick,draw=red,fill=rey] (C3.-70) to[bend left=110,looseness=30] (C3.-110);
                    %
                    \draw [-,rex,line width=2mm] ([xshift=-4mm,yshift=-4mm]C2.-135) to[bend right=30] ([xshift=-4mm,yshift=-4mm]C3.-135);
                    %
                    \filldraw [-,shorten <=2pt,shorten >=2pt,semithick,draw=red,fill=rex] (C2.-115) to[bend left=110,looseness=30] (C2.-155);
                    \filldraw [-,shorten <=2pt,shorten >=2pt,semithick,draw=red,fill=rex] (C3.-115) to[bend left=110,looseness=30] (C3.-155);
                }
            \end{center}
            (ii): The outer two carbon atoms are $sp^3$. The inner two carbon atoms are $sp$.\par
        \end{proof}
        \item \ce{CH2CCHCH3}.
        \begin{proof}[Answer]
            (i) / (iii):
            \begin{center}
                \chemfig{@{C1}C(-[:120]H)(-[:-120]H)=@{C2}C=@{C3}C(-[2]H)-C(-[2]H)(-[6]H)-H}
                \chemmove{
                    \draw [-,rey,line width=2mm] ([xshift=4mm,yshift=4mm]C2.45) to[bend left=30] ([xshift=4mm,yshift=4mm]C3.45);
                    \draw [-,rex,line width=2mm] ([yshift=5mm]C1.90) to[bend left=30] ([yshift=5mm]C2.90);
                    \draw [-,rey,line width=2mm] ([yshift=-5mm]C1.-90) to[bend right=30] ([yshift=-5mm]C2.-90);
                    %
                    \filldraw [-,shorten <=2pt,shorten >=2pt,semithick,draw=red,fill=rey] (C2.65) to[bend left=110,looseness=30] (C2.25);
                    \filldraw [-,shorten <=2pt,shorten >=2pt,semithick,draw=red,fill=rey] (C3.65) to[bend left=110,looseness=30] (C3.25);
                    %
                    \filldraw [-,shorten <=2pt,shorten >=2pt,semithick,draw=red,fill=rex] (C1.110) to[bend left=110,looseness=30] (C1.70);
                    \filldraw [-,shorten <=2pt,shorten >=2pt,semithick,draw=red,fill=rex] (C2.110) to[bend left=110,looseness=30] (C2.70);
                    %
                    \filldraw [-,shorten <=2pt,shorten >=2pt,semithick,draw=red,fill=rey] (C1.-70) to[bend left=110,looseness=30] (C1.-110);
                    \filldraw [-,shorten <=2pt,shorten >=2pt,semithick,draw=red,fill=rey] (C2.-70) to[bend left=110,looseness=30] (C2.-110);
                    %
                    \draw [-,rex,line width=2mm] ([xshift=-4mm,yshift=-4mm]C2.-135) to[bend right=30] ([xshift=-4mm,yshift=-4mm]C3.-135);
                    %
                    \filldraw [-,shorten <=2pt,shorten >=2pt,semithick,draw=red,fill=rex] (C2.-115) to[bend left=110,looseness=30] (C2.-155);
                    \filldraw [-,shorten <=2pt,shorten >=2pt,semithick,draw=red,fill=rex] (C3.-115) to[bend left=110,looseness=30] (C3.-155);
                }
            \end{center}
            (ii): The first and third carbon atoms from the left are $sp^2$. The second carbon atom from the left is $sp$. The right carbon atom is $sp^3$.\par
        \end{proof}
    \end{enumerate}
\end{enumerate}




\end{document}